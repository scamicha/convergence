\documentclass[manuscript]{aastex} 

\newcommand{\ie}{{\it i.e.}}
\newcommand{\cf}{{\it cf.}}
\newcommand{\eg}{{\it e.g.}}
\newcommand{\etal}{{\it et al.}}
\newcommand{\rd}{{\rm d}}
\newcommand{\ag}{\mbox{ \raisebox{-.8ex}{$\stackrel{\textstyle >}{\sim}$} }}
\newcommand{\al}{\mbox{ \raisebox{-.8ex}{$\stackrel{\textstyle <}{\sim}$} }}
\newcommand{\vdag}{(v)^\dagger}
\newcommand{\myemail}{tomsc@astro.indiana.edu}
\newcommand{\mso}{$M_{\odot}$ }
\newcommand{\rso}{$R_{\odot}$ }
\newcommand{\msoyr}{$M_{\odot}/{\rm year}$}

\usepackage{natbib}
\usepackage{color}
\usepackage{graphicx}
\usepackage{epstopdf}
\usepackage{grffile}
\usepackage{amsmath, amsthm, amssymb}
\usepackage{amssymb}


\citestyle{aa}

% \slugcomment{Not to appear in Nonlearned J., 45.}

\shorttitle{Effective Disk Alphas}
\shortauthors{Michael et al.}

%
\begin{document}


\title{Convergence Studies on Mass Transport in Disks with Gravitational Instabilities.
 \\
I. The Constant Cooling Case}
\author{Scott Michael}
\affil{Astronomy Department, Indiana University, Bloomington, IN 47405}
\email{scamicha@astro.indiana.edu}

\author{Thomas Y. Steiman-Cameron}
\affil{Astronomy Department, Indiana University, Bloomington, IN 47405}
\email{tomsc@astro.indiana.edu}

\author{Richard H. Durisen }
\affil{Astronomy Department, Indiana University, Bloomington, IN 47405}
\email{durisen@astro.indiana.edu}

\and

\author{Aaron Boley}
\affil{Department of Astronomy, University of Florida, Gainesville, FL 32611}
\email{aaron.boley@gmail.com}

\begin{abstract}

We conducted a convergence study of the azimuthal resolution in 3D hydrodynamic simulations of a protostellar disk susceptible to gravitational instabilities (GIs). 
The disk was primarily studied during its asymptotic state, when
heating and cooling are in balance. The cooling was treated as a constant global cooling time.
Four simulations were conducted, identical except for the number of azimuthal grid 
points, $l_{max} = $ 64, 128, 256 and 512, used in the calculations. 
The strengths of the azimuthal modes, gravitational torques, and the effective \citet{shakura1973} $\alpha$ 
arising from gravitational stresses were determined for each resolution.  We examined the resolution-dependence of each quantity
and evaluated both the converged modal amplitude distribution and the effective value of $\alpha$
as $l_{max} \rightarrow \infty$.   Non-axisymmetric amplitude was found in all modes and 
the amplitude summed over all modes was essentially independent of resolution.
The modal distribution of amplitudes, however, was significantly affected by resolution.  
The size of the azimuthal grid, $l_{max}$,
limits the number of modes that can be accurately measured to $l_{max}/2$.  Higher resolution simulations therefore
have more modes available to them.  Non-axisymmetric structure is present in all available modes.  As a consequence,
higher resolution simulations have a larger fraction of their total amplitude in high-order modes.
High-order modes
are more confined to act on local structure than low-order modes, therefore as the resolution increases and low-order amplitudes decrease the total gravitational stresses decreases as well.
As a result, higher resolution simulations experience 
weaker gravitational torques than lower resolution simulations.   The effective  $\alpha$
also depends upon the magnitude of the stresses.  Thus
$\alpha_{\rm{eff}}$ is also a function of grid resolution, decreasing as  $l_{max}$ increases.
The converged $\alpha_{\rm{eff}}$ is consistent with
predictions from an analytic local theory for thin disks by \citet{gammie2001}, 
but only over many dynamic times when averaged over a substantial volume of the disk.
Results obtained in the $l_{max} = 512$ simulations differ with converged values
by less than 20\%.

\end{abstract}


\section{Introduction}

Detailed three-dimensional hydrodynamic modeling of protostellar disks has provided considerable insight into the physical and thermal states of these disks.  However,  computing resources limit these studies to snapshots spanning timescales much shorter than a disk's evolutionary lifetime. The level of spatial resolution (and hence temporal resolution, due to the Courant condition) used in calculations can play a deterministic role in interpretations of models; while lower resolutions allow more extensive spatial and temporal studies, this benefit can come at the sacrifice of important physics.  Convergence studies provide a means 
of deriving important outcomes without the expense of very high resolution simulations.  

In contrast to sophisticated calculations that include detailed physics, 
approaches exist that replace detailed physics with statistical parameters that encapsulate this physics.  This
allows for calculations covering much longer timeframes.  Both detailed and 
parametric
approaches can play important roles in broadening our understanding of complex nonlinear systems, provided a connection between the detailed physics and statistical parameters can be demonstrated and understood. In what follows, we perform convergence 
tests on hydrodynamical disk simulations to examine the applicability of an $\alpha$-disk 
formulation \citep{shakura1973} to  the
evolution of protostellar disks subject to gravitational instabilities.   

%Here $\alpha$ is a dimensionless parameter that
%provides a measure of viscosity, without providing a detailed specification of the viscous mechanism.

Gravitational instabilities (GIs) can play an important, and at times dominant, role in driving the structural 
evolution of  protostellar disks \citep[for reviews, see][]{durisen2005, durisen2010}. {\bf (We need the Durisen 2010 (2011?)
review article reference)}
Thermal processes play the primary role in regulating the amplitude and outcome of these instabilities \citep{pickett1998,pickett2000,mejia2005,nelson1998,nelson2000a}. A disk's susceptibility to GIs can be parameterized by the Toomre $Q$-parameter \citep{toomre1981};  $Q = c_s \kappa / \pi G \Sigma$, where $c_s$ is the sound speed, $\kappa$ is the epicyclic frequency ($\sim$ the rotation frequency 
$\Omega$ in a nearly Keplerian disk), and $\Sigma$ is the disk surface mass density. For $Q \al$  1.5 to 1.7,  small density perturbations in a disk grow exponentially on a time scale comparable to the rotation period. These perturbations manifest themselves as multi-arm spirals with a predominantly trailing pattern that transports angular momentum outward by gravitational torques \citep{larson1984,boss1984,durisen1986,papaloizou1991, laughlin1994,nelson1998,pickett1998}. 

Numerous studies have shown that thermal physics control GI amplitudes by a {\it balance} of heating and cooling \citep[e.g.,][]{tomley1991,tomley1994,pickett1998,pickett2000, pickett2003,gammie2001,boss2002,rice2003b,mejia2005,boley2006,boley2007, stamatellos2008,cossins2009}.
Heating by GIs tends to increase $c_s$, thus increasing $Q$. If disk heating persists, the instability will be suppressed once $Q$ becomes sufficiently large.  However, heating and cooling can reach a balance at nonlinear wave amplitude. GI-activity can sustain this balance  at a relatively constant,
but unstable, value of $Q$ \citep{paczynski1978, lin1981,goldreich1965}, and 
the disk can exist in a state of quasi-equilibrium. In this state, cooling is 
balanced by heating arising from the inward flow of matter and shocks induced by the GIs
\citep{gammie2001, lodato2004, rice2005, boley2006, cossins2009, vorobyov2010}.

Fully nonlinear hydrodynamic simulations of low-$Q$ disks show that multiple
modes can become unstable in the linear regime \citep{nelson1998, pickett1998, lodato2004,boley2006,cossins2009}.  Even if the initial growth is dominated by one mode, numerous modes usually appear in the nonlinear regime. As a consequence, if a disk achieves a quasi-steady balance between heating and cooling, referred to as the {\it asymptotic state},
then power will exist at all $m$-values resolved by a given numerical method \citep{mejia2005,boley2006}.  
In the asymptotic state, GI-active disks develop overlapping density structures of different strengths, geometries and coherence, producing strongly nonlinear turbulence that pervades the entire disk \citep[e.g.][]{pickett2003, mejia2005,boley2006}.
\citet{gammie2001} refers to this state as {\it gravitoturbulence}. Gravitoturbulence provides a possible bridge between the detailed physics of GIs and the viscous transport parameter $\alpha$.  Here convergence studies are important because 
higher order modes dissipate energy on shorter wavelengths than lower order modes.  The mix of low and high order modes available to a disk is set by the resolution of the simulation. {\bf Multiple issues with this paragraph. Didn't want to rewrite unilaterally. 1) In the first couple sentences the description of mode growth needs some tweaking, many modes appear in the linear regime, and although we have seen amplitude at all m values I don't know that it's a given. 2) I wouldn't describe the structures formed by GI activity as ``strongly nonlinear turbulence''. 3) In reviewing Gammie 2001 it seems that he is actually referring to a turbulent cascade (a fact I hadn't fully appreciated until now) I think it's pretty clear that the effective alpha we see is, in fact, not due to a turbulent cascade. i.e. alpha is driven by low order coherent modes. We should discuss this at some point SM}

\citet{shakura1973} proposed turbulence as the primary source of effective viscosity in gas disks
and parameterized turbulent transport in a steady-state disk with the  parameter $\alpha$.  The source of the 
turbulence in their $\alpha$-disk formulation is unspecified. Although their original work was directed towards X-ray bright
accretion disks around black holes, this heuristic approach has proven fruitful in many astrophysical situations by allowing for analytic solutions and relatively easy numerical modeling over time-spans inaccessible to detailed modeling. In the 
$\alpha$-disk formalism, when the accretion rate $\dot{M}$ is constant in both time and radius, $\dot{M} = 3\pi\alpha c_s \Sigma H$ where $\Sigma$ is the surface density and $H$ represents
the disk scale height. The $\alpha$ formalism makes no assumption about the nature of the angular momentum transport, only that {\it angular momentum transport
and heating are dominated by local processes}. \citet{lin1987,lin1990} were among the first to suggest
that GI-induced transport could be described within a viscous
framework.  However, the  validity of an $\alpha$-disk picture requires self-consistency of local energy dissipation. 
The locality of dissipation in GI-active disks has not been rigorously tested and remains an open question. 

Several studies support the idea that angular momentum transport by GIs in real disks is, in many important respects, an intrinsically global phenomenon and cannot be properly treated by a local $\alpha$-like prescription \citep{laughlin1996, balbus1999, lodato2005, mejia2005, boley2006, cai2008}.  This almost certainly applies to the embedded phase of both high and low-mass stars when infall from the collapsing protostellar cloud feeds mass into the disk at a rapid
rate \citep[e.g.]{yorke1993, laughlin1994, yorke1999,vorobyov2005,vorobyov2006}.  
However, despite the long-distance nature of gravitational torques, theoretical and numerical results suggest that in protostellar disks where heating and cooling are in balance, i.e., disks subject to gravitoturbulence,
 it may be valid to represent GI-induced transport by an {\it effective viscosity}, a local process \citep{gammie2001, lodato2004, rice2005, cossins2009, vorobyov2010}.  In particular, 
an $\alpha$-viscosity may be consistent
with transport in GI active disks for disk to star mass ratio $M_d/M_* \le 0.2$ -- 0.3; GI-active disks
in systems with larger $M_d/M_*$ are dominated by lower-order modes that act in a more global sense and thus cannot be well represented by an $\alpha$-approach
\citep{lodato2004, cossins2009, vorobyov2010}.

\citet{gammie2001} derived an effective alpha for the case of a thin 2D gravitationally unstable disk where the heating caused by GIs was balanced by the local cooling rate.  
Assuming a razor thin gravitationally unstable disk where the heating caused by
GIs was balanced by a local cooling rate of $t_{cool}\Omega$.
Gammie found
\begin{equation}
\alpha_{\rm{eff}} = [\gamma(\gamma - 1) \slantfrac{9}{4}\Omega t_{cool}]^{-1},
\label{eq:gammie}
\end{equation}
where $\gamma$ is the two dimensional adiabatic index and $t_{cool}$ is a cooling parameter. 
The two dimensional index relates
to the three dimensional index, $\Gamma$, by 
$\gamma = (3\Gamma - 1) / (\Gamma + 1)$
for a non-self-gravitating
disk and
$\gamma = 3 - 2/\Gamma $
for a strongly self gravitating disk.   \cite{lodato2004} derived a similar result when considering three dimensional protoplanetary disks.
Motivated by
the results from \cite{laughlin1996}, \cite{gammie2001}, and \cite{lodato2004, lodato2005}, which demonstrated differences in the manifestation of high and low-order, \citet{kratter2008} 
proposed $\alpha_{\rm{eff}} = (\alpha_{short}^2 + \alpha_{long}^2)^{1/2}$. Here $\alpha_{short}$ and $\alpha_{long}$
are functions of $Q$ and $M_d/\left( M_d+M_*\right)$ representing 
the different wavelength regimes expected to dominate for different values of the mass ratio and cooling rate. 
Fully 3D simulations \citep[e.g.][]{lodato2004, cossins2009} suggest that, at least in the case of imposed cooling, $\alpha_{\rm{\rm{eff}}}$ roughly converges to the value predicted by \citeauthor{gammie2001}. 

We address these, and similar questions, using a grid-based, finite-difference 3D hydrodynamics code
with self gravity.
For computational grids of a fixed {\it physical} size,
grids with more elements provide finer meshes, and thus a more accurate solution. 
However, increased accuracy is offset by computational penalties.
The number of equations solved at each time step 
increases as the grid size is increased.  At the same time, the Courant condition 
forces shorter time steps.  A mesh convergence study 
provides a common approach to balancing accuracy and computing resources.
In the following sections we follow the evolution of a disk subject to GIs
using four different
resolutions in the azimuthal direction.  The number of azimuthal grid points used in simulations
may be particularly important, because higher azimuthal resolution 
allows non-axisymmetric amplitude to spread to higher-order modes 
that behave more locally. We test this by examining the amplitude in various 
modes to see how they are affected by the choice of grid.
In the process, 
we report the results of a convergence study designed to determine whether GI-active disks, chaotic by nature, achieve statistical equilibrium If this is the case, we investigate how closely these disks approximate the gravitoturbulence of \citet{gammie2001} and examine the distribution of $\alpha_{\rm{eff}}$. To this end, we follow the evolution of a GI-active disk, subject to constant cooling, in its asymptotic state, where cooling and heating are essentially balanced.  Simulations are performed using four different resolutions in the azimuthal direction.  

The balance of this paper is organized as follows.  Section 2 provides the details of the numerical approach and defines the models.  An analysis of the modal structure of the simulations 
is found in \S3.1, the gravitational torques arising from these modes follows in \S3.2.  Effective converged alphas are 
discussed in \S3.3.  Section 3.4 deals with the degree to which selected grid sizes lead to results consistent with
convergence.  Fragmentation is discussed in \S3.5.  A summary of results is found in \S4.

%We cannot yet follow the entire dynamical evolution of a 
%protoplanetary disk nor
%model all the radiative and particle processes of interest at once in full 3D. However, 
%by following GI simulations to their asymptotic states, when these exist, we
%can extrapolate the consequences over much longer times (e.g., Boley et al 2006).



\section{Hydrodynamical Simulations}

Simulations were conducted with the three-dimensional hydrodynamics code used by the Indiana University Hydrodynamics Group in several previous studies \citep{pickett1998, pickett2000, mejiaphd2004, mejia2005, boley2006, cai2008}.  This code uses a second-order, explicit, Eulerian scheme to solve Poisson's equation, an equation of state, and the equations of hydrodynamics in conservative form on a uniform cylindrical grid.  The code assumes mirror symmetry about the equatorial plane. It includes both self-gravity and artificial viscosity; the latter serves to 
mediate shocks. Source and flux terms \citep{norman1986} are computed separately in an explicit, second-order time integration \citep{albada1982,christodoulou1991,yangphd1992}.

We assume an equation of state $P = (\gamma -1)\epsilon$, where $P$ is the pressure, $\epsilon$ is the internal energy density, and $\gamma$ is the ratio of specific heats, given here by $\gamma = 5/3$.  The model disks are cooled by decreasing their internal energy according to the prescription $d\epsilon/dt = \epsilon / t_{cool}$, where $t_{cool}$ is the global cooling time in Outer Rotation Periods (ORP).  For the grid resolutions used here, one ORP is defined as the rotational period at radial zone 200.

%\subsection{Models}

Because this is a polytropic disk with an idealized cooling time, only the ratio of disk to stellar mass, given here by 0.153, is specified.  In what follows, quantities specified with real physical units are derived assuming a 1 $M_\odot$ central star and a 0.153  $M_\odot$ disk. In this case, the ORP is defined at $r = 33$ AU and one ORP $\approx$ 303 years.  Models have $t_{cool} = 2$ ORP. The disk has an initial surface density $\Sigma \sim r^{ -1}$, and initial inner and outer radii of 2.3 and 40 AU, respectively.  

Initial conditions for the disk's structure and thermodynamic state were set using an equilibrium star plus disk model generated using a modified \citet{hachisu1986} self-consistent field (SCF) relaxation method \citep{pickett1996,pickett2003,mejiaphd2004,mejia2005,cai2006}. Here  the density and angular momentum distributions are iteratively solved, using the specified equation of state,  until convergence is achieved. This  equilibrium disk was then seeded with small 0.01\% $\delta\rho/\rho$ random perturbations to allow GIs to grow.  The resultant marginally unstable disk served as the initial disk for the simulation.

The initially unstable disk model passed through several phases of evolution \citep[see also][]{pickett2003, mejia2005}.  The initial disk is unstable to the growth of non-axisymmetric structures. During the initial axisymmetric phase, 
which lasts several ORPs, the disk cools and contracts, the contraction is small radially but dramatic in the vertical direction. Around 3-4 ORPs the instabilities begin grow to non-linear amplitudes. When this condition is met, the disk undergoes a strong burst of GIs that predominantly manifest themselves in one or two discrete global spiral modes.  The disk expands violently during the burst phase, producing a significant rearrangement of the disk's mass distribution on a timescale of a few ORPs. This is then followed by a period of several ORPs where heating temporarily washes out some of the non-axisymmetry in the disk.  Finally, following this transition phase, the disk settles into a quasi-steady, long-lived asymptotic state of sustained GI-activity over a large part of the disk, with an overall balance of heating and cooling.

Simulations were run with four different angular resolutions, with grid sizes in $(r,\phi,z)$ of  $(j_{max}, k_{max}, l_{max}) =$ (512,64,64), (512,64,128), (512,64,256), and (512,64,512). Because of the computational overhead associated with running each simulation starting from $t = 0$, the $l_{max} =$ 128 simulation was run through the axisymmetric, burst and transition phases. Simulations using $l_{max} =$ 64, 256, and 512 all begin by interpolating the $l_{max} =$ 128 simulation to higher or lower azimuthal resolution at 9.6 ORPs, near the end of the transition
phase. Although this strategy saves a large amount of computational time, it limits the scope of the analysis to the asymptotic phase. By 12 ORPs, all four disks have transitioned to the asymptotic state.  All simulations were 
run through $\sim 18$ ORPs.

Figure \ref{fig:Final_Q} shows the Toomre $Q$, as a function of radius, for all four simulations at $t \sim$ 18 ORP.  Table \ref{tbl:ams} lists $Q_{avg}$, the value of $Q$ of each of the asymptotic disks, time-averaged from 12 -- 19 ORP and spatially averaged over 10 -- 40 AU.  From $\sim$ 12 -- 50 AU, $Q \sim 1.0$ -- 1.4 and thus the disk is subject GIs over this full range.  Interior to $\sim$ 27 AU, the value of $Q$ shows no dependence on
the azimuthal resolution.  From 27 -- 50 AU, lower resolution simulations display lower values of $Q$. {\bf I suppose 27 AU was chosen by eye? Or was there some other analysis. I vote to toss these last two sentences as I'm not convinced w/o a time averaged quantity. Unless this observation is used in further reasoning in the discussion...SM} 


%%%%%%%%%%%% RESULTS & DISCUSSION %%%%%%%%%%%%%%%%

\section{Results and Discussion}
	
In the asymptotic state, GI-active disks develop complex density structures arising from the superposition of
modes in the disk.  Since the number of modes accessible to the disk depends on the azimuthal resolution, we expect that higher resolution simulations will exhibit non-axisymmetric amplitude in higher-order modes inaccessible at lower resolution, and these higher-order modes will 
behave more locally. We test this by examining the amplitude in various modes to see how they are affected by the choice of grid.

\subsection{Modes}
	
Figure \ref{fig:DensityPlots}, shows mid-plane and radial densities  at $t \sim 18$ ORPs for the $l_{max} =$ 64, 128, 256, and 512 simulations. It is readily apparent that the character of the non-axisymmetric density structures differ with 
resolution. One can clearly see that the lower resolution simulations, i.e. $l_{max} = 64$ and 128, have primarily low-order structures with $m = 2$ and 3 dominating. In contrast, the higher resolution simulations have much more fine structure, the modes are much more washed out, and the low-order modes no longer dominate.
The lower resolution (smaller $l_{max}$) simulations display longer wavelength spiral structures, with larger amplitudes and more coherence than those seen in higher resolution simulations. Higher angular resolution allows non-axisymmetric structures to grow in higher-order azimuthal modes, as demonstrated by a Fourier decomposition of the density. The strengths of these modes are given by their global Fourier amplitudes,

\begin{equation}
A_m = \frac{(a_m^2 + b_m^2)^{1/2}}{\pi\int\rho_o \varpi {\rm d}\varpi {\rm d}z},
\end{equation}
where
\begin{subequations}
\begin{align}
a_m &= \int \rho \cos(m\phi)\varpi {\rm d}\varpi {\rm d}z {\rm d}\phi,\\
b_m &=\int \rho \sin(m\phi)\varpi {\rm d}\varpi {\rm d}z {\rm d}\phi.
\end{align}
\end{subequations}


Here  $\rho_o$ is the axisymmetric component of the density. In what follows, $A_m$ is averaged over time to suppress fluctuations on the dynamic time scale due to the chaotic nature of GIs in the asymptotic  phase.
In particular, $\langle A_{tot} \rangle$ will represent the power over the radial range 10 -- 40 AU, integrated from $ m = 2$ to $l_{max}/2$ and averaged over the time range 12 -- 19 ORPs. The Fourier amplitudes cannot be accurately measured at resolutions smaller than $l_{max}/2$ \citep{shannon1984}, therefore $A_m$ is limited to modes $m \le l_{max}/2$.  
Because the central star is artificially fixed to the grid center, the dynamics of the $m=1$ mode may not
be accurately treated. For this reason, it is excluded from $\langle A_{tot} \rangle$.

{\bf (Note to all: Throughout the paper I use integrals in places where we 
formally have used summations.  I note that this is a common practice, though
not strictly correct.  I find the use of integrals more elegant, but am willing to
revert to summations if people push for it.)}

Previous studies have shown that disks in the asymptotic state have power at all resolvable $m$-values  \citep{lodato2004,mejia2005,boley2006,cossins2009}.  Table \ref{tbl:ams} lists $\langle A_{tot} \rangle$ along with $\langle A_{2-7} \rangle  /  \langle A_{tot} \rangle$,  the fraction of the total power falling into low-order ($m = 2$ -- 7) modes, for each resolution. We identify $m=2-7$ as the low-order modes that are considered to be global modes, again 
$m=1$ is excluded due to the fixed central star. This determination was made by considering the radial range over 
which the mode is most effective in transporting angular momentum, i.e. from the inner Linblad resonance to the outer 
Linblad resonance, and comparing it to the disk scale height. These quantities are roughly equal for $m=7$, for lower 
order modes the radial range exceeds the scale height.
While the total power is approximately independent of resolution, $\langle A_{tot} \rangle = 1.90 \pm 0.07$, 
the distribution of this power shows marked differences. In particular, the fractional power in lower order modes decreases monotonically with increasing resolution.  Over three quarters of $\langle A_{tot} \rangle$ resides
in modes $m =$ 2 -- 7 for the $l_{max} = 64$ simulation, while approximately half resides in these modes when $l_{max} =$ 512.

The shift of power from lower to higher order modes as the resolution of the computational grid is increased arises from the fact that, while $\langle A_{tot} \rangle$ is approximately independent of $l_{max}$,  the number of degrees of freedom available for this power, $l_{max}/2$, increases linearly with the resolution. Thus amplitude is naturally spread from lower order to higher order modes as the available $m$-values increase. Indeed, even if the disk is dominated by one or two low-order modes at the time of outburst, this power cascades to higher-order modes \citep[for example][]{laughlin1997, laughlin1998,laughlin1996}. {\bf Haven't had a chance to check these refs but I'm reluctant to call what we see a cascade of power. I suggest striking this sentence. SM}


Given how the non-axisymmetric amplitude distribution depends on the azimuthal resolution, we must examine what resolution is sufficient to properly capture the evolution of these disks. Table \ref{tbl:ams} shows that while the fractional power in lower order modes decreases with each doubling of $l_{max}$, the amount of change in going from $l_{max} = 256$ to 512 
is significantly less than the change going from $l_{max} = 64$ to 128, with a strong suggestion that this diminution will not proceed much further with higher azimuthal resolution. Indeed, a  Richardson extrapolation \citep{press1992} of the tabulated values of  $\langle A_{2-7} \rangle / \langle A_{tot} \rangle$ suggests a limiting value of  $\langle A_{2-7} \rangle / \langle A_{tot} \rangle \sim$ 0.42 for increasingly higher resolutions. This is displayed graphically in figure \ref{fig:Power_Extrap}, where the fractional power in lower order modes is plotted against the  normalized grid 
spacing for each value of $l_{max}$. The grid spacing  is normalized such that spacing of the $l_{max}$ = 512 grid
equals unity.  Convergence occurs as the normalized spacing goes to zero. The Richardson extrapolation value  of $\langle A_{2-7}\rangle  / \langle A_{tot}\rangle  = 0.42$ is plotted at a normalized grid spacing  equal to zero. 
%The majority of the modal power in a converged disk resides in higher-order modes. 
%These modes dissipate on shorter wavelengths than low-order modes and hence act more locally.

\subsection{Gravitational Torques}

Low-order spiral modes in $\Delta\rho / \rho$  have longer wavelengths that, if coherent in radius, have longer lever 
arms to produce torques, while high-order modes have relatively short wavelengths which, particularly if they 
lack coherence can cancel each other out, and produce more localized effects. 
The finding that the fractional power in high-order modes increases
with resolution, while the total power stays the same, leads to
 the expectation that the gravitational torques in the disk will decrease as $l_{max}$ is increased. This can be tested by examining the torque contributions from  various Fourier modes.

The torque {\bf C} on a cylindrical surface of the disk at radius $\varpi$ can be obtained by integrating the stress tensor $T$ over the surface of the cylinder \citep{lyndenbell1972}, i.e.,
\begin{equation}
{\rm {\bf C}} = \int {\bf r} \times T \cdot {\rm d}S.
\end{equation}
If the stress tensor includes only gravitational stresses, the surface integral can be replaced with the volume integral
\begin{equation}
{\rm \bf C} = \int \rho {\bf r} \times \nabla \Phi {\rm d}V,
\end{equation}
where $\Phi$ is the gravitational potential.  Here we are interested only in the $z$-component of torque,  
\begin{equation}
\label{torque1}
{\rm \bf C}_Z = \int \rho \frac{\partial\Phi} {\partial \phi} {\rm d}V,
\end{equation}
as only this component drives mass and angular momentum transfer. The torque contribution from each mode $m$ can be calculated by replacing $\rho$ in equation \eqref{torque1} with the density distribution reconstructed from a single Fourier component
\begin{equation}
\rho_m = a_{\phi m} \cos(m\phi) + b_{\phi m} \sin(m\phi),
\end{equation}
where $a_{\phi m} =\slantfrac{1}{\pi} \int \rho \cos(m\phi){\rm d}\phi$ and $b_{\phi m} =\slantfrac{1}{\pi} \int \rho \sin(m\phi){\rm d}\phi$, and only the gravitational potential produced by the mass distribution given by $\rho_m$ is included in $\Phi$.

Figure \ref{fig:torquearray} displays 
 time-averaged torques summed over a number of low-order modes, $\sum_1^n \langle {\rm \bf C}_{Z(n)} \rangle$, $n = $ 1, 2, 3, 4, and 5, and $\langle {\rm \bf C}_{Ztot} \rangle$,
the total time-averaged gravitational torque arising from modes 1 thru $l_{max}/2$, for four different time-frames,
as a function of radius and azimuthal resolution. 
The scale of the $y$-axis in each subplot is the same, with the exception of the $l_{max} = 64$ simulation, 
where the scale is doubled.  
Because the star's position is artificially fixed at the center of the computational grid, 
torques arising from the $m=1$ mode,
typically 5-10\% of the total torque when averaged over the radial range of the figure,
should not be considered realistic.  
\citet{michael2010} found that freeing the star changes the torques by 10-20\% at most, 
thus the difference between torques arising from the $m=1$ mode with the star fixed and free to move are small compared to the differences found when comparing azimuthal resolution.

We note two results of figure \ref{fig:torquearray}: 1) the total gravitational torque is smaller for higher resolution simulations, and 2) the total torque is dominated by low-order modes for all azimuthal resolutions. Both reflect the fact that while a higher fraction of the total power in larger $l_{max}$ simulations is in higher order modes, these modes make only minimal contributions to the gravitational torque. As resolution increases, non-axisymmetric amplitude is shifted from low-order global modes, which produce relatively large gravitational torques, to high-order local modes, which tend to cancel each other when integrated over the whole disk. 

{\bf (Note to all: Figure 5 shows torques
time-averaged over for four different time-frames.  I created multiple frames because I wanted to see if torques were 
diminishing with time, as might be expected if the disk has not achieved its asymptotic state.  
Dick has told me he only believes the 12-18 ORP result - Dick, I hope 
I'm not misquoting you. Should we go with multiple panels representing different time-frames or 
go solely with the 12 - 18 orp panel?
} 

{\bf I still think the scale in figure 5 is off by a factor of 2 probably due to the fact that you've used a 0.5 $M_\odot$ star. I also vote for a single 12-18 ORP panel. SM}

\subsection{Effective Alphas}

%An $\alpha$-framework treats disks as essentially two-dimensional objects subject to local energy dissipation.  Therefore, in order to compare our results with predicted and published results, we must extract two-dimensional information from our three-dimensional disks.

{\bf I've made some significant changes to this text from Tom's original. Specifically in the first, fourth, and fifth paragraphs. Please look over and make sure this fits in properly. SM}

After obtaining the gravitational stresses in the disk, one can compute an effective $\alpha$. 
Following \cite{gammie2001} and \citet{lodato2004}, 
\begin{equation}
\label{alpha1}
\alpha_{\rm{eff}}(\varpi) = \left| \frac{{\rm d} \ln \Omega}{{\rm d} \ln \varpi} \right|^{-1} 
\frac {T_{\varpi\phi}^{grav} + T_{\varpi\phi}^{Reyn}}
{\Sigma c_s^2 },
\end{equation}
where $\Omega$ is the azimuthally averaged rotation speed, $T_{\varpi\phi}^{grav}$ and $T_{\varpi\phi}^{Reyn}$ represent the  gravitational and Reynolds stress tensors, respectively, $\Sigma$ the surface density,
and $c_s$ the sound speed. In comparison to \citeauthor{gammie2001}'s 2D models and \citeauthor{lodato2004}'s highly flattened smoothed particle hydrodynamics models, the models presented here have a significant vertical extent. Because of this, the sound speed can vary dramatically from the mid-plane to the low density regions, so it was unclear as to the best way to determine a representative sound speed. In order to include contributions to the sound speed from the entire vertical extent, the denominator $\Sigma c_s^2$ was evaluated as $\int_z \rho c_s^2 {\rm d}z$.
 
%Gammie found
%\begin{equation}
%\alpha_{eff} = [\gamma(\gamma - 1) \slantfrac{9}{4}\Omega t_{cool}]^{-1}
%\end{equation}
%where $\gamma$ is the two dimensional adiabatic index and $t_{cool}$ is a cooling parameter. 
%The two dimensional index relates
%to the three dimensional index, $\Gamma$, by 
%$\gamma = (3\Gamma - 1) / (\Gamma + 1)$
%for a non-self-gravitating
%disk and
%$\gamma = 3 - 2/\Gamma $
%for a strongly self gravitating disk.
  

Numerical uncertainties render Reynolds stresses difficult to determine accurately. 
This difficulty is mitigated by the fact that 
previous studies have found Reynolds stresses to be small relative to gravitational stresses -- see, 
for example, figure 5 of \cite{lodato2004} and figure 10 of \cite{boley2006} -- suggesting that this term
can be safely neglected in the evaluation of $\alpha_{\rm{eff}}$.   For this reason, 
here we omit
$T_{\varpi\phi}^{Reyn}$ from the calculation of $\alpha_{\rm{eff}}$. 
We note that \cite{gammie2001}  found
Reynolds and gravitational stresses contributed equally to $\alpha_{\rm{eff}}$ in his study.  However,
his was a local, shearing-sheet, 2D study.  As such, it was unable to capture global effects, such
as those seen in the cited 3D studies.  


The Newton stresses of equation \eqref{alpha1} can be evaluated using the gravitational torque of equation \eqref{torque1}, i.e.,
\begin{equation}
\label{newtonstress}
\begin{split}
 T_{\varpi\phi}^{grav}\left( \varpi \right) & = \frac{1}{2\pi\varpi^2}C_Z\\
                                & = - \frac{1}{2\pi\varpi^2} \int \rho\frac{\partial\Phi}{\partial\phi} {\rm d}V.
\end{split}
\end{equation}
Figure \ref{fig:alpha_v_radius} shows $\alpha_{\rm{eff}}$, time-averaged from 12 to 18 ORPs, calculated using equations \eqref{alpha1} and \eqref{newtonstress},
for the four grid resolutions.  For comparison, 
$\alpha_{\rm{eff}}$ predicted by equation \eqref{eq:gammie} from \citet{gammie2001} are plotted for disks.  The lower  Gammie curve applies for the strongly self-gravitating limit while the upper curve 
represents the non-self-gravitating limit.   The $l_{max} = 512$ curve is consistent with these two limits.  However, 
the  $l_{max} = 512$ curve does not represent a converged model.
To determine whether the converged value of $\alpha_{\rm{eff}}$ is consistent with predictions, we examine 
the spatially (10 -- 40 AU) and temporally (12 -- 19 ORP) averaged  $\alpha_{avg}$,
provided in table \ref{tbl:ams}, for each value of  $l_{max}$.    A Richardson extrapolation applied to these values  
converges to $\alpha_{avg} = 0.017$ (over the same radial  and time range)
as the grid spacing goes to zero.   This falls squarely
between the $\alpha_{avg}$ values for the Gammie curves 0.014 (strongly self-gravitating) and 0.027 (non-self-gravitating).

At first glance, these results suggest strong agreement with Gammie and thus strong support for dissipative
locality and GIs acting as a local mechanism overall.  However, several cautionary notes are in order. It is important to remember that the bulk of the gravitational torque, and therefore the $\alpha_{\rm{eff}}$, is generated by low order modes that are coherent over tens of AU. As table \ref{tbl:ams} shows, even at the highest resolution, and lowest relative strength of low-order modes, the $A_{2-7}$ modes account for half of the total amplitude. In order to accurately measure and characterize the contribution from these modes, one must consider a significant fraction of the disk. Furthermore, simulations of disks with a large radial extent (e.g. 100s of AU \citep{boley2009}) show coherent low order modes over a significant fraction of the disk. In order to determine the radial extent and strength of such modes one should consider the entire range of radii over which the disk is susceptible to instabilities i.e. $Q \lesssim 1.5$.  

In order to predict an $\alpha_{\rm{eff}}$, \citeauthor{gammie2001}'s formula, presented in equation \eqref{eq:gammie}, also requires knowledge of the cooling rate. This is generally not known {\it a priori} for a given disk. Moreover, the cooling rate may not be easily parametrized with a local or global prescription over the entire disk. Simulations with realistic radiative physics \citep{boley2006} have shown that cooling rates can vary substantially as a function of both radius and time in a GI active disk. In fact, it is these areas where the cooling rates change, leading to more or less GI activity, where many interesting phenomena may occur. 

In addition, at any given radius, 
stresses, and hence $\alpha_{\rm{eff}}$, are highly variable with time.
Figure \ref{fig:alphavar} shows the time-variability of $\alpha_{\rm{eff}}$, 
normalized to its average value between at $t = 14$ and 18 ORP, at seven radial locations.   
Also shown is the orbital period at each radius, provided by the length of the line at the end of each subplot. 
Fluctuations in $\alpha_{\rm{eff}}$
on the order of 100\% or more are seen at all radii
on  time scales range from fractions to integral multiples of the dynamic (orbital) time.  The temporal averaging
of figure \ref{fig:alpha_v_radius} washes out this local variability.
Thus,
even in this highly GI-active region, the predictions of Gammie are only valid over many dynamic times 
when averaged over a substantial volume of the disk.

\subsection{Convergence and $l_{max}$}

The results above permit an assessment of appropriate azimuthal grids for the
problem under consideration and similar problems. The $l_{max} = 512$ simulation yields a value 
for $\alpha_{avg}$ of 0.020. This overestimates the converged value of $\alpha_{avg}
= 0.017$ by $\sim 18$\%. The $l_{max} = 512$ simulation also overestimates the fractional
power in the high-order ($m = 2$ -- 7) modes by essentially the same amount. However, it should be noted that the fractional difference between successive doubling of the resolution has decreased for $l_{max} = 512$. From $l_{max} =64$ to 128 the fractional difference in $\alpha_{avg}$ is 36\%. This decreases to 26\% for the jump from $l_{max} = 256$ to 512. Furthermore, the converged value represents infinite azimuthal resolution, which is, of course, impossible to achieve in a fixed grid scheme. It is unclear as to what variation of $\alpha$ between two values of $l_{max}$ would represent convergence as $\alpha$ itself is a dynamically varying quantity, see figure \ref{fig:alphavar}. 

Another consideration is the effect of $r$ and $z$ resolutions on the outcome of the simulation. In this study we have chosen to focus on azimuthal resolution because it has the greatest effect on resolving the GI structures responsible for producing the gravitational stresses. However, as one increases the azimuthal resolution the radius at which the cells have roughly equal sides increases as well. Ideally, this should occur near the radial range where GIs are most active. For $l_{max} = 512$, $\delta r = \delta \phi r$ at $\approx 13.5$ AU, which is near the radius where GIs are active.

Both the gravitational torques and effective alphas are functions of the
gravitational stresses within the disk.  If the Reynolds stresses are, in fact, negligible relative to the gravitational stresses,
then the errors in the calculated torques and effective alphas that stem from limited azimuthal resolution 
should be the same. In this case, the  $l_{max} = 512$ simulation should also be expected to yield
gravitational torques that are $\sim 18$\% larger than the converged torques.  For comparison, the
errors in the $l_{max} = 256$ simulations would exceed 50\%. For many problems of interest,
results obtained using $l_{max} = 512$ should provide reasonable agreement with results obtained 
with higher resolution. 

It is also worth noting that the overestimation of the effective $\alpha$ is more than likely dependent on the total strength of non-axisymmetric amplitude $\langle A_{tot} \rangle$, which may be smaller depending on the physics of a simulation. For example, \citet{boley2006} found weaker overall GI activity when considering realistic radiative physics. A similar study of the effect of azimuthal resolution in disks cooled with a realistic scheme will be presented in a follow-up paper. 

{\bf Substantially changed this section, might still need some work. SM} 

\subsection{Fragmentation and Effective Alphas}

\noindent
{\bf (Aaron, this section is for you to write or toss, your decision.  The current text should be thought of as a filler only)}

%In an $\alpha$ disk formalism the mechanism giving rise to the transport of angular momentum also acts to dissipate energy. Lodato \& Rice (2005) postulated that a disk in a state of gravitoturbulence would have locally balanced heating and cooling rates. If the heating is due to local dissipation, then $\alpha$ and $t_{cool}$ are related by
%\begin{equation}
%\alpha = \left| \frac{{\rm d}\ln \Omega}{{\rm d} \ln \varpi} \right|^{-2}
%\frac{1}{\gamma(\gamma - 1) t_{cool}\Omega}.
%\end{equation}
%\citet{rice2005} concluded that for a given $t_{cool}$  there exists some maximum stress, quantified by $\alpha_{max}$, which if exceeded would cause a disk to fragment. Their predicted value $\alpha_{max} \approx 0.06$ is shown as a horizontal dotted line in figure \ref{fig:alpha_v_radius}.  Note that both the $l_{max} = $ 64 and 128 simulations exceed this limiting value over some radial range, yet neither displayed fragmentation. For cases where $t_{cool}$  varies with time, \citet{clarke2007} revised this limit to $\alpha_{max} = 0.12$. 

\section{Summary}

\noindent
{\bf (All: My style of writing papers is to write the summary after everything is completed.  Therefore, I have left this
section blank.)}


\acknowledgements
Please add any acknowledgements here.

\bibliographystyle{apj}
\bibliography{general}


\newpage


\begin{deluxetable}{cccccc}
\tablecolumns{5} 
\tablewidth{0pc} 
\tablecaption{$Q$, Fourier Amplitudes, and $\alpha$} 
\tablehead{ 
   \colhead{$l_{max}$} 
& \colhead{$Q_{avg}$}   
& \colhead{$\langle A_{tot} \rangle$}    
& \colhead{$\langle A_{2-7} \rangle / \langle A_{tot} \rangle$} 
& \colhead{$\alpha_{avg}$} \\ 
} 
\startdata
64   & 1.26 & 1.96 &  0.76 & 0.069 \\
128 & 1.24 & 1.91 &  0.62 & 0.044 \\
256 & 1.39 & 1.83 &  0.54 & 0.026 \\
512 & 1.48 & 1.95 &  0.49 & 0.020 \\
\enddata 
\caption{
All quantities are time-averaged over 12 -- 19 ORP.
$Q_{avg}$ and $\alpha_{avg}$ are also spatially
averaged over 10 -- 40 AU.
}
\label{tbl:ams}
\end{deluxetable}
\newpage

\begin{figure}
\plotone{figures/Figure1.eps}
\caption{Azimuthally averaged mid-plane Toomre Q distribution versus radius for each of the four resolutions at t $\sim$ 18 ORPs}
\label{fig:Final_Q}
\end{figure}

\begin{figure}
\plotone{figures/Figure2.eps}
\caption
{
Mid-plane disk densities at $t \sim 18$ ORPs for simulations with 64, 128, 256 and 512 azimuthal grid elements.
The simulations differ only in the number of  azimuthal zones used in the calculations.  Densities out of the plane, along an azimuthal cut through the disk, are displayed below the face-on views.  The color scale is logarithmic, and axis units are AU. 
}
\label{fig:DensityPlots}
\end{figure}
\newpage

\begin{figure}
\plotone{figures/Figure3.eps}
\caption
{
The logarithm of the amplitudes for the global Fourier components of the non-axisymmetric density structure plotted for different $m$-values. Notice that the components with the highest global amplitudes are those with low $m$-values.  
}
\label{fig:Am_vs_log_m}
\end{figure}

\begin{figure}
\plotone{figures/Figure4.eps}
\caption
{
Convergence of fractional power.  The $x$-axis represents
the azimuthal ``spacing'' between grid points,  normalized such that spacing of the $l_{max}$ = 512 grid
equals unity.
The $y$-axis displays the fraction of the total power in the low-order ($m = $ 2--7) Fourier modes. 
Also plotted is the Richardson Extrapolation for the asymptotic value  
$\langle A_{2-7} \rangle / \langle A_{tot} \rangle = 0.43$ as the grid spacing goes to zero.  
}
\label{fig:Power_Extrap}
\end{figure}

\begin{figure}
\plotone{figures/Figure5.eps}
\caption
{
Torque Array}
\label{fig:torquearray}
\end{figure}


\begin{figure}
\label{fig:alpha_v_radius}
\plotone{figures/Figure6.eps}
\caption
{Effective $\alpha$'s for the $t_{cool} =$ 2 ORP, $\gamma = 5/3$, $\Sigma \sim r^{ -1}$ disk, averaged between 12 and 18 ORPs, for four different different azimuthal grid resolutions, given by the number of azimuthal grid-points $l_{max}$. The dashed lines show the predictions of Gammie (2001) for $\alpha$ (upper: strongly self-gravitating limit, lower: non-self-gravitating limit), while the dashed line at $\log(\alpha) = -1.22$ represents the maximum stress a disk can maintain without fragmenting (Rice et al. 2005). {\bf suggest removing the Rice line if it is not discussed in the paper.} 
}
\end{figure}
\newpage

\begin{figure}
\plotone{figures/Figure7.eps}
\caption
{Alpha variations.}
\label{fig:alphavar}
\end{figure}


\end{document}

