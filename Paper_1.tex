\documentclass[manuscript]{aastex} 

\newcommand{\ie}{{\it i.e.}}
\newcommand{\cf}{{\it cf.}}
\newcommand{\eg}{{\it e.g.}}
\newcommand{\etal}{{\it et al.}}
\newcommand{\rd}{{\rm d}}
\newcommand{\ag}{\mbox{ \raisebox{-.8ex}{$\stackrel{\textstyle >}{\sim}$} }}
\newcommand{\al}{\mbox{ \raisebox{-.8ex}{$\stackrel{\textstyle <}{\sim}$} }}
\newcommand{\vdag}{(v)^\dagger}
\newcommand{\myemail}{tomsc@astro.indiana.edu}
\newcommand{\mso}{$M_{\odot}$ }
\newcommand{\rso}{$R_{\odot}$ }
\newcommand{\msoyr}{$M_{\odot}/{\rm year}$}

\usepackage{natbib}
\usepackage{color}
\usepackage{graphicx}
\usepackage{epstopdf}
\usepackage{grffile}
\usepackage{amsmath, amsthm, amssymb}
\usepackage{amssymb}
\usepackage{ulem}
\graphicspath{{figures/}}

\definecolor{magenta}{rgb}{1.0,0.0,0.85}

\newcommand{\ACBc}[1]{{\color{red} #1}}
\newcommand{\TSCc}[1]{{\color{magenta} #1}}
\newcommand{\SMc}[1]{{\color{blue} #1}}

\citestyle{aa}

% \slugcomment{Not to appear in Nonlearned J., 45.}

\shorttitle{Effective Disk Alphas}
\shortauthors{Michael et al.}

%
\begin{document}


\title{Convergence Studies of Mass Transport in Disks with Gravitational Instabilities.
 \\
I. The Constant Cooling Time Case}
\author{Scott Michael}
\affil{Astronomy Department, Indiana University, Bloomington, IN 47405}
\email{scamicha@astro.indiana.edu}

\author{Thomas Y. Steiman-Cameron}
\affil{Astronomy Department, Indiana University, Bloomington, IN 47405}
\email{tomsc@astro.indiana.edu}

\author{Richard H. Durisen }
\affil{Astronomy Department, Indiana University, Bloomington, IN 47405}
\email{durisen@astro.indiana.edu}

\and

\author{Aaron C. Boley}
\affil{Department of Astronomy, University of Florida, Gainesville, FL 32611}
\email{aaron.boley@gmail.com}

\begin{abstract}

We conduct a convergence study of a protostellar disk, subject to a constant 
global cooling time and susceptible to gravitational instabilities (GIs), at
a time when heating and cooling are roughly balanced. Our goal is to 
determine the gravitational torques produced by GIs, the level to which transport can
be represented by a simple alpha-disk formulation, and to examine 
fragmentation criteria. Four simulations are conducted, identical except for the number
of azimuthal computational grid points used. A Fourier decomposition of 
nonaxisymmetric density structures in  $\cos(m\phi)$, $\sin(m\phi)$ is performed to evaluate 
the amplitudes $A_m$ of these structures. The $A_m$, gravitational torques, and
the effective \citet{shakura1973} $\alpha$  arising from gravitational stresses
are determined for each resolution. We find nonzero $A_m$ for all $m$-values and
that $A_m$  summed over all $m$ is essentially independent of resolution. Because
the number of measurable $m$-values is limited to half the number of azimuthal
grid points, higher-resolution simulations have a larger fraction of their total
amplitude in higher-order structures. These structures act more locally than
lower-order structures. Therefore, as the resolution increases the total 
gravitational stress decreases as well, leading higher-resolution simulations to experience
weaker average gravitational torques than lower-resolution simulations. The 
effective $\alpha$ also depends upon the magnitude of the stresses, thus
$\alpha_{eff}$ also decreases with increasing resolution. Our converged
$\alpha_{eff}$ is consistent with predictions from
an analytic local theory for thin disks by Gammie (2001), but only over many
dynamic times when averaged over a substantial volume of the disk.

\end{abstract}


\section{Introduction}

Detailed three-dimensional hydrodynamic modeling of protostellar disks has provided considerable insight into the physical and thermal states of these disks.  However,  computing resources limit these studies to durations spanning timescales much shorter than a disk's evolutionary lifetime. The level of spatial resolution, and hence temporal resolution due to the Courant condition, can play a deterministic role in interpretations of calculations; while lower resolutions allow more extensive spatial and temporal studies, this benefit can come at the sacrifice of important physics.  Convergence studies provide a means 
of deriving important outcomes while assessing the resolution needed to determining these outcomes.  

In contrast to sophisticated calculations that include detailed physics, 
approaches exist that replace detailed physics with statistical parameters that encapsulate this physics.  This
allows for calculations covering much longer time frames.  Both detailed and 
parametric
approaches can play important roles in broadening our understanding of complex nonlinear systems, provided a connection between the detailed physics and statistical parameters can be demonstrated and understood. In what follows, we perform convergence 
tests on hydrodynamical disk simulations to examine the applicability of an $\alpha$-disk 
formulation \citep{shakura1973} to  the
evolution of protostellar disks subject to gravitational instabilities.   

Gravitational instabilities (GIs) can play an important, and at times dominant, role in driving the structural 
evolution of  protostellar disks \citep[for reviews, see][]{durisen2005, durisen2007}.
Thermal processes play the primary role in regulating the amplitude and outcome of these instabilities \citep{pickett1998,pickett2000,pickett2003,nelson1998,nelson2000a, mejia2005}. A disk's susceptibility to GIs can be parameterized by the Toomre $Q$-parameter \citep{toomre1981};  $Q = c_s \kappa / \pi G \Sigma$, where $c_s$ is the sound speed, $\kappa$ is the epicyclic frequency ($\sim$ the rotation frequency 
$\Omega$ in a nearly Keplerian disk), and $\Sigma$ is the disk surface mass density. For $Q \al$ 1.7,  small density perturbations in a disk grow exponentially on a timescale comparable to the rotation period \citep{durisen2007}. These perturbations manifest themselves as multi-arm spirals with a predominantly trailing pattern that transports angular momentum outward by gravitational torques \citep{larson1984,boss1984,durisen1986,papaloizou1991, laughlin1994,nelson1998,pickett1998}. 

Numerous studies have shown that thermal physics control GI amplitudes by a {\it balance} of heating and cooling \citep[e.g.,][]{tomley1991,tomley1994,pickett1998,pickett2000, pickett2003,gammie2001,boss2002,rice2003b,mejia2005,boley2006,boley2007b, stamatellos2008,cossins2009}.
Heating by GIs tends to increase $c_s$, thus increasing $Q$. If disk heating persists, the instability will be
suppressed once $Q$ becomes sufficiently large.  On the other hand, radiative cooling tends to oppose the increase in
$Q$ by lowering $c_s$. In this way, heating and cooling can reach a rough overall balance at nonlinear wave amplitude. GI-activity can sustain this balance  at a relatively constant,
but unstable, value of $Q$ \citep{paczynski1978, lin1981,goldreich1965}, and 
the disk can exist in a state of quasi-equilibrium. In this state, cooling is 
balanced by heating arising from the inward flow of matter and shocks induced by the GIs
\citep{gammie2001, lodato2004, rice2005, boley2006, cossins2009, vorobyov2010}.

%\ACBc{[ACB: I've been thinking about the whole heating-by-pdV-to-stabilize-disks thingy.  I am not so sure it is true! Let's 
%consider a purely isentropic compression.  In this case, $Q= (\gamma K)^{1/2}\Sigma^{\left(2-\gamma\right)/2}\kappa/(\pi G)$. 
%A compression STILL destabilizes the disk.  One needs to change $K$. Have I set up a straw man by restricting myself to 
%thinking of an isentropic compression? I do not think so because pdV should not change entropy. ]}

We note here a caution on semantics.  GIs in disks commonly manifest themselves  as non-axisymmetric density structures whose strengths can be characterized 
by a Fourier decomposition of the density in terms of $\sin m\phi$ and $\cos m\phi$, where 
$m$ is the order of the Fourier term and $\phi$ is the usual 
cylindrical coordinate.  While discrete coherent spiral waves seen in these  disks may well be correctly thought of as eigenmodes, particularly 
at low $m$-values, the analysis necessary to confirm that these structures are, in fact, truly modes is generally lacking.
Power in a specific $m$-value does not imply the existence of that mode, nor does it necessarily represent the strength of a mode that truly exists.  Indeed, a disk with a {\it single} $m = 2$ eigenmode growing to nonlinear amplitudes will exhibit power at all even values of $m$.  A disk with {\it two} nonlinear eigenmodes of
$m = 2$ and 3 will exhibit power at all $m$-values.  
For this reason, in the balance of this paper we will avoid referring to the Fourier terms as modes, but rather will refer to them as Fourier structures or components.   The term {\it mode} will be reserved for those cases where eigenmodes are known, or strongly suspected, to exist.

Fully nonlinear hydrodynamic simulations of low-$Q$ disks show that multiple
modes can become unstable in the linear regime \citep{nelson1998, pickett1998, lodato2004,boley2006,cossins2009}.  Even if the initial growth is dominated by one mode, numerous modes usually appear in the nonlinear regime. As a consequence, if a disk achieves a quasi-steady balance between heating and cooling, referred to as the {\it asymptotic state},
then power will exist at all $m$-values resolved by a given numerical method \citep{mejia2005,boley2006}.  
In the asymptotic state, GI-active disks develop overlapping density structures of different strengths, geometries and coherence that pervade the entire disk \citep[e.g.,][]{pickett2003, mejia2005,boley2006}.
%\citet{gammie2001} refers to this state as {\it gravitoturbulence}. Gravitoturbulence 
These structures provide a possible bridge between the detailed physics of GIs and the viscous transport parameter $\alpha$
\citep{gammie2001}.  Here convergence studies are important because 
higher-order modes dissipate energy on shorter wavelengths than lower-order modes; the mix of low and high-order azimuthal density structures
available to a disk is set by the resolution of the simulation.

\citet{shakura1973} proposed turbulence as the primary source of effective viscosity in gas disks
and parameterized turbulent transport in a steady-state disk with the  parameter $\alpha$ \citep[see also][]{pringle1981}.  In this case,  the kinematic
viscosity is defined by $\nu = \alpha c_s H$, where  $c_s$ is the sound speed and $H$ is the disk height.
The source of the 
turbulence in their $\alpha$-disk formulation is unspecified. Although their original work was directed towards X-ray bright
accretion disks around black holes, this heuristic approach has proven fruitful in many astrophysical situations by allowing for analytic solutions and relatively easy numerical modeling over time-spans inaccessible to detailed modeling. 
%In the  $\alpha$-disk formalism, when the accretion rate $\dot{M}$ is constant in both time and radius, 
%$\dot{M} = 3\pi\alpha c_s \Sigma H$ where $\Sigma$ is the surface density and $H$ represents
%the disk scale height. 
The $\alpha$ formalism makes no assumption about the nature of the angular momentum transport, only that {\it angular momentum transport
and heating are dominated by local processes}. \citet{lin1987,lin1990} were among the first to suggest
that GI-induced transport could be described within a viscous
framework.  However, the  validity of an $\alpha$-disk picture requires self-consistency of local energy dissipation. 
The locality of dissipation in GI-active disks has not been thoroughly tested and remains an open question \citep{lodato2004,lodato2005,boley2006}. 

In fact, several studies support the idea that angular momentum transport by GIs in real disks is, in many important respects, an intrinsically global phenomenon and cannot be properly treated by a local $\alpha$-like prescription \citep{laughlin1996, balbus1999, lodato2005, mejia2005, boley2006, cai2008}.  This almost certainly applies to the embedded phase of both high and low-mass stars when infall from the collapsing protostellar cloud feeds mass into the disk at a rapid
rate \citep[e.g.,][]{yorke1993, laughlin1994, yorke1999,vorobyov2005,vorobyov2006,kratter2010}.  
However, despite the long-distance nature of gravitational torques, theoretical and numerical results suggest that in protostellar disks where heating and cooling are in rough balance, 
 it may be valid to represent GI-induced transport by an {\it effective viscosity}, as if it were a local process \citep{gammie2001, lodato2004, rice2005, cossins2009, vorobyov2010}.  In particular, multiple studies suggest that
an effective $\alpha$-viscosity may be consistent
with transport in GI-active disks for a disk to star mass ratio $M_d/M_* \le 0.2$ -- 0.3; GI-active disks
in systems with larger $M_d/M_*$ are dominated by lower-order modes that act in a more global sense and thus cannot be well represented by an $\alpha$-approach
\citep{lodato2004, cossins2009, vorobyov2010}. 
We note, however, that the these studies all assumed
a {\it local cooling time}, $t_{cool}\Omega =$ constant, 
where $t_{cool}$ is defined by the rate at which cooling decreases the disk's internal energy density
$\epsilon$ by $d\epsilon/dt = \epsilon / t_{cool}$.  In contrast, we and our collaborators use either a {\it global cooling time},
$t_{cool} = $ constant \citep[e.g.,][this paper]{pickett2003, mejia2005}, or use realistic opacities \citep[e.g.,][]{mejiaphd2004, cai2006, cai2008, boley2006, boley2007b,  boley2009} which lead to something in between $t_{cool} = $ constant  and
$t_{cool}\Omega =$ constant. The review by \citet{durisen2007} discusses some consequences of these various
choices. There is a tendency for $t_{cool}\Omega =$ constant to bias a disk toward behavior like that of a
steady-state $\alpha$-disk.

\citet{gammie2001} derived an effective $\alpha$ for the case of a thin gravitationally unstable disk where the heating
caused by GIs was balanced by the local cooling rate \citep[see also][]{pringle1981}.  
Assuming a local cooling time of $t_{cool}\Omega =$ constant,  Gammie found
\begin{equation}
\alpha_{\rm{eff}} = [\gamma(\gamma - 1) \slantfrac{9}{4}\Omega t_{cool}]^{-1},
\label{eq:gammie}
\end{equation}
where $\gamma$ is the two-dimensional adiabatic index. 
The two-dimensional index relates
to the three-dimensional adiabatic index, $\Gamma$, by 
$\gamma = (3\Gamma - 1) / (\Gamma + 1)$
for a non-self-gravitating
disk and
$\gamma = 3 - 2/\Gamma $
for a strongly self gravitating disk.   \cite{lodato2004} derived a similar result when considering three-dimensional protoplanetary disks.
Fully 3D simulations \citep[e.g.,][]{lodato2004, cossins2009} suggest that, at least in the case of imposed 
$t_{cool}\Omega = $ constant cooling, $\alpha_{\rm{\rm{eff}}}$ roughly converges to the value given by equation (\ref{eq:gammie}). 

In what follows, we address these and related questions using a grid-based, finite-difference 3D hydrodynamics code with 
self-gravity to follow the evolution of a protostellar disk with a constant cooling time and subject to GIs.
We examine how grid resolution affects
the amplitude distribution of azimuthal density structures arising from GIs, the gravitational torques arising from these structures,
the effective $\alpha$, and the extent to which transport can be represented by an $\alpha$-disk formulation.
%For computational grids of a fixed {\it physical} size,
%grids with more elements provide finer meshes, and thus a more accurate solution. 
%However, increased accuracy is offset by computational penalties.
%The number of equations solved at each time step 
%increases as the grid size is increased.  At the same time, the Courant condition 
%forces shorter time steps.  
A mesh convergence study 
provides a common approach to balancing accuracy and computing resources.
It permits an assessment of whether or not convergence occurs and what resolution is required.
%allowing us to determine if convergence occurs and what is the required resol
In the following sections, we follow the evolution of a disk subject to GIs
using grids with four different
resolutions in the azimuthal direction.  The number of azimuthal elements used in simulations
may be particularly important, because higher azimuthal resolution 
allows power in non-axisymmetric density structures to spread to higher-order modes 
that behave more locally. 
Although full convergence testing eventually requires investigation of the radial and vertical resolution in our
cylindrical grid, these tests are expensive and will be left to a future paper.  We suspect that the azimuthal
direction is most important to test first because it is non-axisymmetry that produces the torques leading
to mass and angular momentum transport.
We test this by examining the amplitude in various 
$m$-values to see how they are affected by the choice of grid.
In the process, 
we report the results of a convergence study designed to determine whether GI-active disks, chaotic by nature, achieve statistical equilibrium.   If this is the case, we investigate how closely these disks approximate equation (1) from
 \citet{gammie2001} and examine the distribution of $\alpha_{\rm{eff}}$. 

The remainder of this paper is organized as follows.  Section 2 provides the details of the numerical approach and defines the models.  An analysis of the azimuthal density structures of the simulations 
is found in \S3.1, the gravitational torques arising from these structures follows in \S3.2.  Effective converged alphas are 
discussed in \S3.3.  Section 3.4 deals with the degree to which selected grid-sizes lead to results consistent with
convergence.  Fragmentation is discussed in \S3.5.  A summary of results is found in \S4.


\section{Hydrodynamical Simulations}

Simulations were conducted with the three-dimensional hydrodynamics code used by the Indiana University Hydrodynamics Group in several previous studies \citep{pickett1998, pickett2000, mejiaphd2004, mejia2005, boley2006, cai2008}.  This code uses a second-order, explicit, Eulerian scheme to solve Poisson's equation, an ideal gas equation of state, and the equations of hydrodynamics in conservative form on a uniform cylindrical grid.  The code assumes mirror symmetry about the equatorial plane. It includes both self-gravity and artificial viscosity; the latter serves to 
mediate shocks. Source and flux terms \citep{norman1986} are computed separately in an explicit, second-order time integration \citep{albada1982,christodoulou1991,yangphd1992}.

Initial conditions for the disk's structure and thermodynamic state were set using an equilibrium star plus disk model generated by a modified \citet{hachisu1986} self-consistent field relaxation method \citep{pickett1996,pickett2003,mejiaphd2004,mejia2005,cai2006}. Here  the density and angular momentum distributions are iteratively solved, using the specified equation of state,  until convergence is achieved. This  equilibrium disk was 
given random cell-to-cell $\sim10^{-4}$ density perturbations to seed growth of  GIs.  The resultant marginally unstable disk served as the initial disk for the simulation.

Because this is a polytropic disk with an idealized cooling time, only the ratio of disk to stellar mass, given here by
0.153, is specified.  In what follows, quantities specified with real physical units are derived assuming a 1 $M_\odot$
central star and a 0.153  $M_\odot$ disk. In this case, the outer rotation period (ORP) is defined at $r = 33$ AU and
one ORP $\approx$ 214 years.  The disk has an initial surface density $\Sigma \sim r^{ -1}$, with initial inner and
outer radii of 2.3 and 40 AU, respectively. The construction of these initial equilibrium disk models is discussed in
\citet{pickett2003} and \citet{mejia2005}.

 We assume an equation of state $P = (\gamma -1)\epsilon$, where $P$ is the pressure, $\epsilon$ is the internal energy
 density, and $\gamma$ is the ratio of specific heats, given here by $\gamma = 5/3$. As discussed in \citet{boley2007a},
 the rotation states of $H_2$ are probably not excited in the cold outer parts of the disk ($> 10$ AU); so $\gamma =
 5/3$ is an appropriate choice. The model disks are cooled by decreasing their internal energy according to the prescription $d\epsilon/dt = \epsilon / t_{cool}$, where $t_{cool}$ is the global cooling time, which is set to 2 ORP everywhere. 
 
The initially unstable disk model passed through several phases of evolution \citep[see also][]{pickett2003, mejia2005}.  
During the initial axisymmetric phase, 
which lasts several ORP, the disk cools and contracts.  The contraction is small radially but dramatic in the vertical direction. Around 3 to 4 ORP, the instabilities begin to grow to nonlinear amplitudes, and the disk 
then undergoes a strong burst of GIs that predominantly manifest themselves in a few discrete global spiral modes.  The disk expands violently during the burst phase, producing a significant rearrangement of the disk's mass distribution on a timescale of a few ORP. This is then followed by a period of several ORP where heating temporarily washes out some of the non-axisymmetry in the disk.  Finally, following this transition phase, the disk settles into a quasi-steady, long-lived asymptotic 
phase of sustained GI-activity over a large part of the disk, with an overall balance of heating and cooling.

Simulations were run with four different angular resolutions, with grid sizes in cylindrical coordinates $(r,\phi,z)$ of  $(j_{max}, k_{max}, l_{max}) =$ (512,64,64), (512,64,128), (512,64,256), and (512,64,512). 
For reference, the radius $r = 33$ AU at which the ORP is defined is at radial cell $j = 200$.  The radial grid extends to 512
to accomodate disk expansion during outburst.
Because of the computational overhead associated with running each simulation starting from $t = 0$, the $l_{max} =$ 128 simulation was run through the axisymmetric, burst and transition phases. Simulations using $l_{max} =$ 64, 256, and 512 all begin by a linear interpolation of the $l_{max} =$ 128 simulation to higher or lower azimuthal resolution at 9.6 ORP, near the end of the transition
phase. Although this strategy saves a large amount of computational time, it limits the scope of the analysis to the asymptotic phase. By 12 ORP, all four disks appear to have transitioned to the asymptotic state,
e.g., they all display roughly steady $Q(r)$ profiles starting at $\sim 12$ ORP.  All simulations were 
run through $\sim 18$ ORP.

Figure \ref{fig:Final_Q} shows the Toomre $Q$, as a function of radius, for all four simulations at $t \sim$ 18 ORP.  
%Table \ref{tbl:ams} 
Table 1 lists $Q_{avg}$, the arithmetic mean of $Q$ for each of the asymptotic disks, time-averaged from 12 -- 18 ORP and spatially averaged over 10 -- 40 AU.  From $\sim$ 12 -- 50 AU, $Q \sim 1.0$ -- 1.4 and thus the disk is subject to GIs over this full range.  

%%%%%%%%%%%% RESULTS & DISCUSSION %%%%%%%%%%%%%%%%

\section{Results and Discussion}

\subsection{Non-axisymmetric Structure}
	
In the asymptotic state, GI-active disks develop complex density structures arising from the superposition of non-axisymmetric
modes in the disk.  Since the number of modes accessible to the disk depends on the azimuthal resolution \citep{shannon1984}, we expect that higher resolution simulations will exhibit non-axisymmetric amplitude in higher-order modes inaccessible at lower resolution, and these higher-order modes will 
behave more locally. 
	
Figure \ref{fig:DensityPlots}, shows mid-plane and radial densities  at $t \sim 18$ ORP for the $l_{max} =$ 64, 128, 256, and 512 simulations. It is readily apparent that the character of the non-axisymmetric density structures differ with 
resolution. One can clearly see that the lower resolution simulations, i.e., $l_{max} = 64$ and 128, have primarily low-order structures with $m = 2$ and 3 dominating. In contrast, the higher resolution simulations have much more fine structure, 
and the low-order $m$-values no longer dominate.  
The lower resolution (smaller $l_{max}$) simulations display longer wavelength spiral structures, with larger amplitudes and more coherence than those seen in higher resolution simulations. Higher angular resolution allows non-axisymmetric structures to grow in higher-order azimuthal $m$-values, as demonstrated by a Fourier decomposition of the density. The strengths of these components are given by their global Fourier amplitudes,

\begin{equation}
A_m = \frac{(a_m^2 + b_m^2)^{1/2}}{\pi\int\rho_o \varpi {\rm d}\varpi {\rm d}z},
\end{equation}
where
\begin{subequations}
\begin{align}
a_m &= \int \rho \cos(m\phi)\varpi {\rm d}\varpi {\rm d}z {\rm d}\phi,\\
b_m &=\int \rho \sin(m\phi)\varpi {\rm d}\varpi {\rm d}z {\rm d}\phi.
\end{align}
\end{subequations}
Here  $\rho_o$ is the axisymmetric component of the density. In what follows, $A_m$ is averaged over time to suppress fluctuations on the dynamic timescale due to the chaotic nature of GIs in the asymptotic  phase.
In particular, $\langle A_{tot} \rangle$ will represent the power over the radial range 10 -- 40 AU, summed from $ m = 2$ to $l_{max}/2$ and averaged over the time range 12 -- 18 ORP. The Fourier amplitudes cannot be measured at resolutions smaller than $l_{max}/2$, therefore $A_m$ is limited to modes $m \le l_{max}/2$.  
Because the central star is artificially fixed to the grid center, the $m=1$ mode may not
be accurately treated. For this reason, it is excluded from $\langle A_{tot} \rangle$.  As discussed below in \S3.2,
we expect this exclusion to have minimal impact on the results of this work.


Previous studies have shown that disks in the asymptotic state have power at all resolvable $m$-values  \citep
{lodato2004,mejia2005,boley2006,cossins2009}.  This can be seen in
figure 3, where $\log \langle A_m\rangle$ is displayed as a function of $\log(m)$.  For all resolutions, components with the highest global amplitudes are those with low $m$-values; $m=2$ through 6 are labeled.  
In what follows, the term ``low-order''  will refer to Fourier components with $m <  8$.
Table 1 lists $\langle A_{tot} \rangle$ along with $\langle A_{2-7} 
\rangle  /  \langle A_{tot} \rangle$,  the fraction of the total power falling into low-order ($m = 2$ -- 7) azimuthal structures, 
for each resolution. 
We identify $m=2-7$ as representing amplitude in the low-order modes, i.e., coherent structure that is
global in character.   Again 
$m=1$ is excluded due to the fixed central star. This determination of $m$-range
was made by considering the radial range over 
which the mode is most effective in transporting angular momentum, i.e., from the inner Linblad resonance to the outer 
Linblad resonance, and comparing it to the disk scale height. These quantities are roughly equal for $m=7$; for 
lower-order modes the radial range exceeds the scale height. The vertical dotted line in figure 3 corresponds with $m=7$. 
While the total power is approximately independent of resolution, $\langle A_{tot} \rangle \sim 1.90$, 
the distribution of this power amongst $m$-values
shows marked differences. In particular, the fractional power in lower-order $m$-values decreases monotonically with increasing resolution.  Over three quarters of $\langle A_{tot} \rangle$ resides
in $m =$ 2 -- 7 for the $l_{max} = 64$ simulation, while approximately half resides in these $m$-values when $l_{max} =$ 512.

The shift of power from lower-order to higher-order azimuthal structure as the resolution of the computational grid is increased arises from the fact that, while $\langle A_{tot} \rangle$ is approximately independent of $l_{max}$,  the number of degrees of freedom available for this power, $l_{max}/2$, increases linearly with the resolution. Thus amplitude is naturally spread from lower-order to higher-order as the available $m$-values increase. 
Indeed, even if the disk is dominated by a few low-order modes at the time of outburst, this power spreads to all
$m$-values in the nonlinear regime \citep[for example][]{laughlin1997, laughlin1998,laughlin1996}. 

Given how the non-axisymmetric amplitude distribution depends on the azimuthal resolution, we must examine what resolution is sufficient to properly capture the evolution of these disks. Table 1 shows that while the fractional power in 
lower-order $m$-values
decreases with each doubling of $l_{max}$, the amount of change in going from $l_{max} = 256$ to 512 
is significantly less than the change going from $l_{max} = 64$ to 128, with a strong suggestion that this diminution will not proceed much further with higher azimuthal resolution. Indeed, a  Richardson extrapolation \citep{press1992} of the tabulated values of  $\langle A_{2-7} \rangle / \langle A_{tot} \rangle$ suggests a limiting value of  $\langle A_{2-7} \rangle / \langle A_{tot} \rangle \sim$ 0.43 for increasingly higher resolutions. This is displayed graphically in figure \ref{fig:Power_Extrap}, where the fractional amplitude in lower-orders is plotted against the  normalized grid 
spacing for each value of $l_{max}$. The grid spacing  is normalized such that spacing of the $l_{max}$ = 512 grid
equals unity. 
Convergence occurs as the normalized spacing goes to zero. The Richardson extrapolation value  
of $\langle A_{2-7}\rangle  / \langle A_{tot}\rangle  = 0.43$ is plotted at a normalized grid spacing  equal to zero. 
 
%\ACBc{[ACB: This is good, but I think we can do better by: (1) getting a handle on what we think the statistical variation in the %$A$ values is, and (2) using that to get an extrapolated value that has some likelihood value.]} \TSCc{[TSC: I think this would %be overkill.  It would involve a bit of work and I think we would not gain any additional insight into the problem by doing this %work. Added note - During my talk with Aaron, he seemed adamant about this.  Dealing with point would take a bit
%of time and effort.  I am not convinced it is worthwhile, but I defer to the rest of the group.]}


\subsection{Gravitational Torques}

Low-order spiral modes in the density distribution have longer wavelengths that, if coherent in radius, have longer lever 
arms to produce torques, while high-order modes have relatively short wavelengths which, particularly if they 
lack coherence, can cancel each other out and produce more localized effects. 
The finding that the fractional power in high-order structure increases
with resolution, while the total power stays the same, leads to
 the expectation that the gravitational torques in the disk will decrease as $l_{max}$ is increased. This can be tested by examining the torque contributions from  various Fourier components.

The torque {\bf C} on a cylindrical surface of the disk at radius $\varpi$ can be obtained by integrating the stress tensor $T$ over the surface of the cylinder \citep{lyndenbell1972, boley2006}, i.e., 
\begin{equation}
{\rm {\bf C}} = \int {\bf r} \times T \cdot {\rm d}S.
\end{equation}
If the stress tensor includes only gravitational stresses, the surface integral can be replaced with the volume integral
\begin{equation}
{\rm \bf C} = \int \rho {\bf r} \times \nabla \Phi {\rm d}V,
\end{equation}
where $\Phi$ is the gravitational potential.  Here we are interested only in the $z$-component of torque,  
\begin{equation}
\label{torque1}
{\rm \bf C}_Z = \int \rho \frac{\partial\Phi} {\partial \phi} {\rm d}V,
\end{equation}
as only this component drives mass and angular momentum transfer. The torque contribution from each mode $m$ can be calculated by replacing $\rho$ in equation \eqref{torque1} with the density distribution reconstructed from a single Fourier component
\begin{equation}
\rho_m = a_{\phi m} \cos(m\phi) + b_{\phi m} \sin(m\phi),
\end{equation}
where $a_{\phi m} =\slantfrac{1}{\pi} \int \rho \cos(m\phi){\rm d}\phi$ and $b_{\phi m} =\slantfrac{1}{\pi} \int \rho \sin(m\phi){\rm d}\phi$, and only the gravitational potential produced by the mass distribution given by $\rho_m$ is included in $\Phi$.
Note that, as defined here, $\rho_m$ is a  function of $\varpi$, $z$ and $\phi$ while $a_{\phi m}$ and
$b_{\phi m}$ are functions of $\varpi$ and $z$.

Figure \ref{fig:torquearray} displays 
 time-averaged torques summed over a number of low-order modes, $\sum_1^n \langle {\rm \bf C}_{Z(n)} \rangle$, $n = $ 1, 2, 3, 4, and 5, and $\langle {\rm \bf C}_{Ztot} \rangle$,
the total time-averaged gravitational torque arising from modes 1 through $l_{max}/2$
as a function of radius and azimuthal resolution. 
The scale of the $y$-axis in each subplot is the same, with the exception of the $l_{max} = 64$ simulation, 
where the scale is doubled.  
Because the star's position is artificially fixed at the center of the computational grid, 
torques arising from the $m=1$ mode,
typically 5-10\% of the total torque when averaged over the radial range of the figure,
should not be considered realistic.  
\citet{michael2010} found that freeing the star changes the torques by about 10-20\%, 
thus the difference between torques arising from the $m=1$ mode with the star fixed and free to move are small compared
to the differences found when comparing azimuthal resolution. In fact, it is expected that the $m =1$ mode would only be
effective at higher disk-to-star mass ratios \citep{shu1990}.

We note two results in figure \ref{fig:torquearray}: 1) the total gravitational torque is smaller for higher resolution simulations and 2) the total torque is dominated by low-order Fourier components for all azimuthal resolutions. Both reflect the fact that while a higher fraction of the total power in larger $l_{max}$ simulations is in higher-order $m$-values, these structures make only minimal contributions to the gravitational torque. As resolution increases, non-axisymmetric amplitude is shifted from 
low-order global components, which produce relatively large gravitational torques, to high-order local structures, which tend to cancel each other when integrated over the whole disk. 


\subsection{Effective Alphas}

After obtaining the gravitational stresses in the disk, one can compute an effective $\alpha$. 
Following \cite{gammie2001}, \citet{lodato2004}, and \citet{boley2006}, 
\begin{equation}
\label{alpha1}
\alpha_{\rm{eff}}(\varpi) = \left| \frac{{\rm d} \ln \Omega}{{\rm d} \ln \varpi} \right|^{-1} 
\frac {T_{\varpi\phi}^{grav} + T_{\varpi\phi}^{Reyn}}
{\Sigma c_s^2 },
\end{equation}
where $\Omega$ is the azimuthally averaged rotation speed, $T_{\varpi\phi}^{grav}$ and $T_{\varpi\phi}^{Reyn}$ represent the  gravitational and Reynolds stress tensors, respectively, $\Sigma$ the surface density,
and $c_s$ the sound speed. In comparison to \citeauthor{gammie2001}'s 2D models, the models studied here have a significant vertical extent. Because of this, the sound speed can vary dramatically from the mid-plane to the low density regions. In order to include contributions to the sound speed from the entire vertical extent, the denominator $\Sigma c_s^2$ was evaluated as $\int_z \rho c_s^2 {\rm d}z$.
 
%Gammie found
%\begin{equation}
%\alpha_{\rm{eff}} = [\gamma(\gamma - 1) \slantfrac{9}{4}\Omega t_{cool}]^{-1}
%\end{equation}
%where $\gamma$ is the two dimensional adiabatic index and $t_{cool}$ is a cooling parameter. 
%The two dimensional index relates
%to the three dimensional index, $\Gamma$, by 
%$\gamma = (3\Gamma - 1) / (\Gamma + 1)$
%for a non-self-gravitating
%disk and
%$\gamma = 3 - 2/\Gamma $
%for a strongly self gravitating disk.
  

Numerical uncertainties render Reynolds stresses difficult to determine accurately. 
This difficulty is mitigated by the fact that 
previous studies have found Reynolds stresses to be small relative to gravitational stresses (see, 
for example, figure 5 of Lodato \& Rice 2004 and figure 10 of Boley et al. 2006).  This  suggests that the Reynolds stress
can be neglected in the evaluation of $\alpha_{\rm{eff}}$.   For this reason, 
here we omit
$T_{\varpi\phi}^{Reyn}$ from the calculation of $\alpha_{\rm{eff}}$. 
We note that \cite{gammie2001}  found
Reynolds and gravitational stresses contributed equally to $\alpha_{\rm{eff}}$ in his study.  However,
his was a local, shearing-sheet, 2D study.  As such, it was unable to capture global effects, such
as those seen in the cited 3D studies.  


The gravitational stresses of equation \eqref{alpha1} can be evaluated using the gravitational torque of equation \eqref{torque1}, i.e.,
\begin{equation}
\label{newtonstress}
\begin{split}
 T_{\varpi\phi}^{grav}\left( \varpi \right) & = \frac{1}{2\pi\varpi^2}C_Z\\
                                & = - \frac{1}{2\pi\varpi^2} \int \rho\frac{\partial\Phi}{\partial\phi} {\rm d}V.
\end{split}
\end{equation}
Figure \ref{fig:alpha_v_radius} shows $\alpha_{\rm{eff}}$, time-averaged from 12 to 18 ORP, calculated using equations \eqref{alpha1} and \eqref{newtonstress},
for the four grid resolutions.  For comparison, 
$\alpha_{\rm{eff}}$ predicted by equation \eqref{eq:gammie} from \citet{gammie2001} are plotted for disks.  The lower  Gammie curve applies for the strongly self-gravitating limit while the upper curve 
represents the non-self-gravitating limit.   The $l_{max} = 512$ curve is consistent with these two limits.  However, 
the  $l_{max} = 512$ curve does not represent a converged model.
To determine whether the converged value of $\alpha_{\rm{eff}}$ is consistent with predictions, we examine 
the spatially (10 -- 40 AU) and temporally (12 -- 18 ORP) averaged  $\alpha_{avg}$,
provided in table 1, for each value of  $l_{max}$. A Richardson extrapolation applied to these values  
converges to $\alpha_{avg} = 0.017$ (over the same radial  and time range)
as the grid spacing goes to zero.   This falls
between the $\alpha_{avg}$ values for the Gammie curves 0.014 (strongly self-gravitating) and 0.027 (non-self-gravitating).

At first glance, these results suggest strong agreement with Gammie and thus strong support for dissipative
locality and GIs acting as a local mechanism overall.  However, several cautionary notes are in order. It is important to remember that the bulk of the gravitational torque, and therefore the $\alpha_{\rm{eff}}$, is generated by low-order 
Fourier components that are coherent over tens of AU.  As table 1 shows, even at the highest resolution, where
the relative strength of low-order structures is smallest, $\langle A_{2-7} \rangle$ accounts for half of the total amplitude.
In order to accurately measure and characterize the contribution from these modes, one must consider a significant fraction of the disk. Even when averaged over 6 ORP, figure 6 shows that $\alpha_{\rm{eff}}$ displays $\sim 100$\% spatial fluctuations
over a few AU.
Furthermore, simulations of disks with a larger radial extent, e.g., 100s of AU \citep{boley2009}, show coherent 
low-order modes over a significant fraction of the disk. 
In order to determine the radial extent and strength of such modes, one should consider the entire range of radii over which the disk is susceptible to instabilities, i.e., $Q \lesssim 1.5$ (figure 1).  In this context,
\cite{balbus1999}
argue that a self-consistent local-disk picture, as required for a proper $\alpha$-disk formulation,
is probably not possible for GI-unstable disks,
except perhaps near the corotation radius of a spiral wave.
For the lowest-order modes in our disk, this is approximately centered at $\sim 25$ AU.

In order to compare an $\alpha_{\rm{eff}}$, \citeauthor{gammie2001}'s formula, presented in equation \eqref{eq:gammie}, also requires knowledge of the cooling time. This is generally not known {\it a priori} for a given disk. Moreover, the cooling time may not be easily parametrized with a local or global prescription over the entire disk. Simulations with realistic radiative physics \citep{boley2006} have shown that cooling times can vary substantially as a function of both radius and time in a GI active disk. In fact, it is these areas where the cooling rates change, leading to more or less GI activity, where many interesting phenomena may occur.
In addition, at any given radius, the
stresses, and hence $\alpha_{\rm{eff}}$, are highly variable with time.
Figure \ref{fig:alphavar} shows the time-variability of $\alpha_{\rm{eff}}$, 
normalized to its average value between $t = 14$ and 18 ORP, at seven radial locations
for times later than 12 ORP.
Here the time axis is normalized to the orbital time.  Because the orbital periods are nearly
Keplerian, at smaller radii the disk is followed for many more dynamical times than at larger
radii.  For ease of comparison, 
only the last 14 orbits are shown for $r = 10$ and 15.
Fluctuations in $\alpha_{\rm{eff}}$
on the order of 100\% or more are seen at all radii
on  timescales characteristic of the local dynamical time, indicating that
this variability about the average is local in nature.  The temporal averaging
of figure \ref{fig:alpha_v_radius} washes out this local variability.
Thus,
even in this highly GI-active region, Gammie's formula only seems valid over many dynamic times 
when averaged over a substantial volume of the disk. 


\subsection{Convergence and $l_{max}$}

The results above permit an assessment of appropriate azimuthal grids for the problem under consideration and
similar problems. The $l_{max} = 512$ simulation yields a value for $\alpha_{avg}$ of 0.020. This
overestimates the extrapolated converged value of $\alpha_{avg} = 0.017$ by $\sim 18$\%. The $l_{max} = 512$
simulation also overestimates the fractional power in the high-order ($m = 2$ -- 7) modes by essentially the
same percentage. However, it should be noted that the fractional difference between successive doubling of the
resolution has decreased for $l_{max} = 512$. From $l_{max} =64$ to 128 the fractional difference in
$\alpha_{avg}$ is 36\%. This decreases to 26\% for the jump from $l_{max} = 256$ to 512. Furthermore, the
converged value represents infinite azimuthal resolution, which is, of course, impossible to achieve in a
fixed grid scheme.
%It is unclear as to what variation of $\alpha$ between two values of $l_{max}$ would represent convergence as $\alpha$ 
%itself is a dynamically varying quantity, see figure \ref{fig:alphavar}. 

%DICK: I made changes here.
In addition to the convergence of $\alpha_{avg}$, table 1 illustrates the convergence of the measured net
inward mass transport rates $\dot{M}$ over the time interval 12 -- 18 ORP. We tabulate the value of
$|\dot{M}|$ spatially averaged over 10 -- 40 AU. The absolute value of the net $\dot{M}$ at each $r$ is taken
prior to the spatial average because $\dot{M}$ varies in sign with $r$. It is the magnitude of the typical net
mass transport rates that we wish to compare. The $|\dot{M}|_{avg}$ measurements show convergence similar to
that of the $\alpha_{avg}$ measurement based on the gravitational torques. This gives us confidence that no
important contributions from Reynolds stresses are being overlooked. Moreover, we have used eq. (22) of
\citet{boley2006} to estimate mass inflow rates based on the measured gravitational torques and find that
these torques are indeed sufficient to account for the measured $\dot{M}$.

Another consideration is the effect of $r$ and $z$ resolutions on the outcome of the simulation. In this study we have chosen to focus on azimuthal resolution because it has the greatest effect on resolving the non-axisymmetric
structures responsible for producing the gravitational stresses. However, as one increases the azimuthal resolution, the radius at which the cells have roughly equal sides increases as well. Ideally, this should occur near the radial range where GIs are most active. For $l_{max} = 512$, $\delta r = r \delta \phi $ at $\approx 13.5$ AU, which is near the radius where GIs are active.

Both the gravitational torques and $\alpha_{\rm{eff}}$ are functions of the
gravitational stresses within the disk.  If the Reynolds stresses are, in fact, negligible relative to the gravitational stresses,
then the percentage errors in the calculated torques and $\alpha_{\rm{eff}}$ that stem from limited azimuthal resolution 
should be the same. In this case, the  $l_{max} = 512$ simulation should also be expected to yield
gravitational torques that are $\sim 18$\% larger than the converged torques.  For comparison, the
errors in the $l_{max} = 256$ simulations would exceed 50\%. For many problems of interest,
results obtained using $l_{max} = 512$ should provide reasonable agreement with results obtained 
with higher resolution. 
Although computationally expensive, simulations with even higher resolution in all directions and codes using different grid geometry
would be worthwhile.

\subsection{Fragmentation and Effective $\alpha$s}

\noindent

Under certain conditions, GIs fail to self-regulate, allowing density perturbations to grow until clumps form from spiral structure (see Durisen et al. 2007 for references).  Under isothermal conditions, fragmentation becomes very likely whenever $Q$ drops below 1.4 
\citep{tomley1991, tomley1994, nelson1998, mayer2002}. When the effective adiabatic index is stiff due to inefficient
cooling, shocks and mass transport can militate against fragmentation, even in disks with initially low $Q$ values
\citep[e.g., ][]{boley2008}. To understand the degree of cooling required to effect clump formation in non-isothermal
disks, \cite{gammie2001} studied the nonlinear behavior of GIs using a series of shearing sheet simulations.  He showed
that  a gravitationally unstable disk is likely to fragment whenever the cooling time is less than half the local
orbital period ($t_{cool}\Omega = 3$).  This work was followed by Rice et al.~(2005), who used global 3D SPH simulations and found that the fragmentation limit was dependent on the adiabatic index.   For disks with $\gamma=5/3$, they found fragmentation 
occurred whenever the local cooling time was less than the local orbital period. For $\gamma=7/5$, the limit was twice the local orbital period.  \cite{rice2005} proposed that this fragmentation threshold was due to a maximum stress that can  be exerted on a disk by gravitational torques.  Using the $\alpha_{\rm{eff}}$ as defined in equation (8), they found that fragmentation could occur whenever $\alpha_{\rm eff}\gtrsim 0.06$ (the dotted line in figure 6).  The difference between the cooling time threshold found by Gammie and the ones found by Rice et al.~are not obviously at odds, as Gammie's study was conducted in 2D and used a 2D adiabatic index of 2.  

Recently, these results have been called into question by \cite{meru2011a}, who suggest that our understanding of the
fragmentation criterion is based on simulations that have yet to converge.  In particular, they find that fragmentation
becomes increasingly likely for longer cooling times when the resolution in their 3D SPH simulations is increased,
giving weight to the claims by, e.g., \cite{boss2009}, that fragmentation is a robust result of disk instability. The
Meru \& Bate study uses a prescribed cooling such that $t_{\rm cool}=\beta\Omega^{-1}$, where $\Omega$ is the local
orbital frequency and $\beta$ is some global constant.  This same cooling prescription was used in the Gammie and Rice
et al.~studies.  The reason for the differences is unclear, and may not be simply a resolution effect (see Lodato \&
Clarke 2011, arXiv:1101.2448).  For example, the \cite{gammie2001} have revealed the most stringent cooling criterion of
all the $t_{cool}$ fragmentation studies, and, due to the local approximation Gammie used, these simulations are also at
very high resolution. Moreover, \cite{mejia2005}, who used grid-based simulations with the same code we use here, and  \cite{boley2008}, who used radiative hydrodynamics simulations, also found results that are reasonably consistent with those of Gammie, Rice et al., and \cite{cossins2010}. 

The present convergence study does not address fragmentation directly, but there are several results that should be noted.  First, although the total power in non-axisymmetric structure is roughly constant throughout the simulations, the average $Q$ value increases with increasing resolution (Table 1).  Because both fast cooling and a low $Q$ are necessary
for fragmentation, this suggests that the disk is becoming less susceptible to clump formation with increasing resolution.   Second, the disks do not show a trend toward denser spiral structure as the resolution is increased which is seen by eye in figure 2, and is quantified by the decrease in $\left<A_{2-7}\right>$.  The high-$m$ modes are increasing in power, but remain smaller than the low-$m$ components.   Third, our results show that as the resolution increases, the disk approaches a self-regulating state that is reasonably well-described, in the time- and space-average sense, by an $\alpha$-disk with the parameterization of \cite{gammie2001}. 

Some caution must be taken when applying our results generally.  The structure of the disk at the time of the onset of GIs may have a large impact on fragmentation.  For example, \cite{boley2009} found that for disks with the same mass and {\it total} mass accretion rate, the disk with the shallower density profile fragmented, while the other did not \citep[see also ][]{meru2011b}.  The disks explored here evolve to a surface density profile of approximately $p=5/2$ in the outer regions for $\Sigma\propto r^{-p}$, while those explored by \citet{meru2011a} have a shallower profile ($p=1$), favoring fragmentation.  In addition to disk structure, the thermal history of the disk also plays a role in the fragmentation threshold.  \cite{clarke2007}  found that the fragmentation cooling limit is sensitive to the way the instability is approached.  When the cooling time is decreased slowly, the fragmentation limit is pushed toward shorter cooling times than if a fast cooling time is suddenly prescribed.  

Overall, the disks simulated here are converging toward a state of self-regulation as the resolution is increased, and do not show signs of becoming more susceptible to fragmentation. 


\section{Summary}

We conducted a convergence study of a protostellar disk subject to gravitational instabilities to
examine the distribution and amplitudes 
of non-axisymmetric density structures arising from GIs 
and the connection of these structures to gravitational torques,
to the level which transport can be represented by a simple $\alpha$-disk formulation,
and to fragmentation.
A 3D hydrodynamics code with self-gravity
was used to follow a disk subject to a constant global cooling time during its asymptotic state, when
heating and cooling are roughly in balance.
Four simulations were conducted, identical except for the number of azimuthal grid 
points, $l_{max} = $ 64, 128, 256, and 512, used in the calculations.

A Fourier decomposition of the non-axisymmetric density structure 
in $\cos(m\phi)$, $\sin(m\phi)$ was performed in order to
characterize the amplitude distribution,  $A_m$,
of azimuthal structures as a function of $m$.  
This and previous studies have shown
that disks in the asymptotic state have power at all resolvable $m$-values.
The maximum value of  $m$ is given by
$m_{max} = l_{max}/2$.  Therefore simulations with larger $l_{max}$ can spread the
power over more Fourier components.
We find that $\langle A_{tot} \rangle$, the power summed over all $m$, is roughly the same
for all resolutions.
However, the distribution of power among the Fourier components
is distinctly affected by $l_{max} $.  
% $A_7$ is approximately equal for all resolutions.
The fractional
amplitude in low-order modes, $\langle A_{2-7} \rangle / \langle A_{tot} \rangle$,
where $\langle A_{2-7} \rangle$ is 
the sum over modes $m = 2$ through 7, 
decreases monotonically with increasing resolution.  Over three quarters of $\langle A_{tot} \rangle$ resides
in $m =$ 2 -- 7 for the $l_{max} = 64$ simulation, while approximately half resides in these $m$-values when $l_{max} =$ 512.
A Richardson extrapolation suggests a limiting value of $\langle A_{2-7} \rangle / \langle A_{tot} \rangle \sim 0.43$ 
as $l_{max}\rightarrow \infty$.
The number of degrees of freedom available for power, $l_{max}/2$, increases linearly with the resolution.  Thus
Fourier amplitudes are naturally spread from lower-order to higher-order as the available $m$-values increase. 

Non-axisymmetric density structures produce gravitational torques, driving radial flows of mass and angular momentum.   While the total torque is dominated by low-order Fourier components for all azimuthal resolutions, the total
time-averaged gravitational torque  is smaller for higher resolution simulations.   This reflects the fact that while a higher fraction of the total power in larger $l_{max}$ simulations is in higher-order Fourier components, these components make only minimal contributions to the gravitational torque. As $l_{max}$ increases, non-axisymmetric amplitude is shifted from low-order global components, which produce relatively large gravitational torques, to high-order local structures, which tend to cancel each other when integrated over the whole disk. 

The effective $\alpha_{\rm{eff}}$ was determined for each simulation under the assumption that
Reynolds stresses are negligible relative to the gravitational stresses, as found in
previous 3D studies.  
As is the case for gravitational torques, $\alpha_{\rm{eff}}$  decreases with increased resolution.  
Derived values of 
$\alpha_{\rm{eff}}$ vs. radius for the highest resolution simulation, $l_{max} = 512$, 
are consistent with
predictions from an analytic local theory for thin disks by \citet{gammie2001}, as is the
converged $\alpha_{\rm{eff}}= 0.017$, time-averaged over the asymptotic phase and spatially averaged 
over the disk. The time and spatially averaged $\alpha_{\rm{eff}}$  
obtained for the $l_{max} = 512$ simulations overestimates the converged value
by less than 20\%.  Gravitational torques for this same model differ with converged values by a
similar amount.
Locally determined values of
$\alpha_{\rm{eff}}$ vary by factors of 2 or 3 throughout the disk on timescales characteristic
of the local dynamical time.  Thus the predictions of Gammie are only valid over many dynamic times when averaged over a substantial volume of the disk.

Although not a goal of this work, our simulations permit
comments on studies on fragmentation and clump formation.
Most recent studies indicate that our disk should not fragment \citep[e.g., ][]{gammie2001, rice2005, mejia2005, boley2008}.   However, some recent studies \citep[e.g., ][]{boss2009, meru2011a}
%Meru & Bate (2011), 
suggest that fragmentation may be a common result of gravitational instabilities in disks.
We find that as the resolution increases, our disk approaches a self-regulating state that can be reasonably well-described, in the time- and space-average sense, by an $\alpha$-disk parameterization. 
The average Toomre $Q$ increases with increasing resolution,
suggesting that the disk is becoming less susceptible to clump formation with increasing resolution,
consistent with the observation that
the disk simulations do not show a trend toward denser spiral structure as the resolution is increased.
Overall, the disks simulated here are converging toward a state of self-regulation as the resolution is increased, and do not show signs of becoming more susceptible to fragmentation. 
 
A future paper will address the issues presented in this work using a convergence study of GI-susceptible disks subject to radiative cooling with realistic opacities.  This will allow us to address the generality of the results cited herein.

\acknowledgements

SM would like to thank NASA for the generous funding of the
NASA Earth and Space Science Fellowship NNX07AU82H. This
research was also supported in part by NASA Origins of Solar
Systems grants NNG05GN11G and NNX08AK36G. The simulations
and analysis herein were conducted on hardware generously
provided by the Indiana University Information Technology Services
and was supported by the National Science Foundation under
Grant no. ACI-0338618l, OCI-0451237, OCI-0535258, CNS-
0521433 and OCI-0504075. This work was also supported in part
by Shared University Research grants from IBM, Inc., to the Indiana
University.


\bibliographystyle{apj}
\bibliography{general}


\newpage

%DICK: I made changes here.
\begin{table}
\begin{center}
\begin{tabular}{cccccc}
\multicolumn{5}{c}{$Q$, Fourier Amplitudes, $\alpha$, and $\dot{M}$}\\ \hline\hline
$l_{max}$ & $Q_{avg}$   & $\langle A_{tot} \rangle$   & $\langle A_{2-7} \rangle / \langle A_{tot} \rangle$ &
$\alpha_{avg}$ & $|\dot{M}|_{avg} ~\mathrm{M_\odot/yr}$\\\hline
64   & 1.26 & 1.96 &  0.76 & 0.069 & 8.9$ \times 10^{-7}$ \\
128 & 1.24 & 1.91 &  0.62 & 0.044 & 3.7$ \times 10^{-7}$ \\
256 & 1.39 & 1.83 &  0.54 & 0.026 & 2.5$ \times 10^{-7}$ \\
512 & 1.48 & 1.95 &  0.49 & 0.020 & 2.4$ \times 10^{-7}$ \\ 
$\infty$ & & &0.43 & 0.017 & \\ \hline
\label{tbl:ams}
\end{tabular}
\caption{ All quantities are time-averaged over 12 -- 18 ORP.
$Q_{avg}$, $\alpha_{avg}$, and $|\dot{M}|_{avg}$  are also spatially
averaged over 10 -- 40 AU.  The entries for ``$\infty$'' are based on Richardson extrapolation,
as explained in the text.}  
\end{center}
\end{table}


%\begin{deluxetable}{cccccc}
 %\tablecolumns{5}
%\tablewidth{0pc} 
%\tablecaption{$Q$, Fourier Amplitudes, and $\alpha$} 
%\tablehead{ 
 %  \colhead{$l_{max}$} 
%& \colhead{$Q_{avg}$}   
%& \colhead{$\langle A_{tot} \rangle$}    
%& \colhead{$\langle A_{2-7} \rangle / \langle A_{tot} \rangle$} 
%& \colhead{$\alpha_{avg}$} \\ 
%} 

%\startdata
%64   & 1.26 & 1.96 &  0.76 & 0.069 \\
%128 & 1.24 & 1.91 &  0.62 & 0.044 \\
%256 & 1.39 & 1.83 &  0.54 & 0.026 \\
%512 & 1.48 & 1.95 &  0.49 & 0.020 \\
%\enddata  
%\label{tbl:ams}
%\caption{
%All quantities are time-averaged over 12 -- 19 ORP.
%$Q_{avg}$ and $\alpha_{avg}$ are also spatially
%averaged over 10 -- 40 AU.
%}
%\end{deluxetable}


\newpage

\begin{figure}
\plotone{Figure1.eps}
\caption{Azimuthally averaged mid-plane Toomre Q distribution versus radius for each of the four resolutions at t $\sim$ 18 ORP}
\label{fig:Final_Q}
\end{figure}

\begin{figure}
\plotone{Figure2.eps}
\caption
{
Mid-plane disk densities at $t \sim 18$ ORP for simulations with 64, 128, 256 and 512 azimuthal grid elements.
The simulations differ only in the number of  azimuthal zones used in the calculations.  Densities out of the plane, along an azimuthal cut through the disk, are displayed below the face-on views.  The color scale is logarithmic, and axis units are AU. 
}
\label{fig:DensityPlots}
\end{figure}
\newpage

\begin{figure}
\plotone{Figure3.eps}
\caption
{
The logarithm of the amplitudes for the global Fourier components of the non-axisymmetric density structure plotted for
different $m$-values. Notice that the components with the highest global amplitudes are those with low $m$-values; $m=2$
through 6 are labeled.  The vertical dotted line corresponds with $m=7$.  For modes with $m < 7$,  the radial range
$\Delta r_{LR}$ from  the inner Linblad resonance to the outer Linblad resonance exceeds the disk scale height; the
$\Delta r_{LR}$ and scale height are roughly equal for $m = 7$.
}
\label{fig:Am_vs_log_m}
\end{figure}

\begin{figure}
\plotone{Figure4.eps}
\caption
{
Convergence of the fractional amplitude of low-order non-axisymmetric structure.  The $x$-axis represents
the azimuthal ``spacing'' between grid points,  normalized such that spacing of the $l_{max}$ = 512 grid
equals unity.
The $y$-axis displays the fraction of the total power in the low-order ($m = $ 2--7) Fourier modes. 
Also plotted is the Richardson Extrapolation for the asymptotic value  
$\langle A_{2-7} \rangle / \langle A_{tot} \rangle = 0.43$ as the grid spacing goes to zero.  
}
\label{fig:Power_Extrap}
\end{figure}

\begin{figure}
\plotone{Figure5.eps}
\caption
{
Total time-averaged gravitational  torques, along with contributions arising from torques
summed over combinations of  $m = $ 1, 2, 3, 4, and 5,
as a function of radius and azimuthal resolution. 
Higher resolution simulations are subject to smaller total torques while the total torque is dominated by low-order $m$-value
components for all azimuthal resolutions.  While a higher fraction of the total power in larger $l_{max}$ simulations is in higher-order $m$-values, these components make only minimal contributions to the gravitational torque. 
}
\label{fig:torquearray}
\end{figure}


\begin{figure}
\plotone{Figure6.eps}
\caption
{Effective $\alpha$'s for the $t_{cool} =$ 2 ORP, $\gamma = 5/3$, $\Sigma \sim r^{ -1}$ disk, averaged between 12 and 18
  ORP, for four different azimuthal grid resolutions, given by the number of azimuthal grid-points $l_{max}$. The
  dashed lines show the $\alpha$ from equation (\ref{eq:gammie}) (upper: strongly self-gravitating limit, lower:
  non-self-gravitating limit), while the dotted line at $\log(\alpha) = -1.22$ represents the maximum stress a disk can
  maintain without fragmenting according to \citet{rice2005}. 
}
\label{fig:alpha_v_radius}
\end{figure}
\newpage

\begin{figure}
\plotone{Figure7.eps}
\caption{
Local value of $\alpha$ at seven radii shown as a function of the local dynamical time.  For radii greater than 20 AU, the plot shows 
the variation from 12 to 18 ORP.  At 10 and 15 AU, the 14 orbit periods immediately prior to 18 ORP are shown.
Dotted lines at an ordinate value of unity represent the average from 14 to 18 ORP at that radius.
Note that the local $\alpha$ displays $\sim$ 100\% variations on the local dynamical time. 
}
\label{fig:alphavar}
\end{figure}


\end{document}

