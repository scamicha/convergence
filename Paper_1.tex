\documentclass[manuscript]{aastex} % one-column, double-spaced document
%\documentclass[preprint2]{aastex}  %% double-column, single-spaced document
% \documentclass[12pt,preprint]{aastex}
% \documentclass[aaspp4]{aastex}
% \documentclass[aas2pp4]{aastex}
% \documentclass[aj_pt4]{aastex}
% \documentclass[apjpt4]{aastex}
% \documentstyle[12pt,aasms4]{article}
% \documentclass[12pt]{article}
%\documentstyle[12pt]{article}
% \tighten
%
%
\newcommand{\ie}{{\it i.e.}}
\newcommand{\cf}{{\it cf.}}
\newcommand{\eg}{{\it e.g.}}
\newcommand{\etal}{{\it et al.}}
\newcommand{\rd}{{\rm d}}
\newcommand{\ag}{\mbox{ \raisebox{-.8ex}{$\stackrel{\textstyle >}{\sim}$} }}
\newcommand{\al}{\mbox{ \raisebox{-.8ex}{$\stackrel{\textstyle <}{\sim}$} }}
\newcommand{\vdag}{(v)^\dagger}
\newcommand{\myemail}{tomsc@astro.indiana.edu}
\newcommand{\mso}{$M_{\odot}$ }
\newcommand{\rso}{$R_{\odot}$ }
\newcommand{\msoyr}{$M_{\odot}/{\rm year}$}

\usepackage{natbib}
\usepackage{color}
\usepackage{graphicx}
\usepackage{epstopdf}
\usepackage{grffile}
\usepackage{amsmath, amsthm, amssymb}
\usepackage{amssymb}


\citestyle{aa}

% \slugcomment{Not to appear in Nonlearned J., 45.}

\shorttitle{Effective Disk Alphas}
\shortauthors{Michael et al.}

%
\begin{document}


\title{Convergence Studies and Effective Alphas for Disks with Gravitational Instabilities.
 \\
I. The Constant Cooling Case}
\author{Scott Michael}
\affil{Astronomy Department, Indiana University, Bloomington, IN 47405}
\email{scamicha@astro.indiana.edu}

\author{Thomas Y. Steiman-Cameron}
\affil{Astronomy Department, Indiana University, Bloomington, IN 47405}
\email{tomsc@astro.indiana.edu}

\author{Richard H. Durisen }
\affil{Astronomy Department, Indiana University, Bloomington, IN 47405}
\email{durisen@astro.indiana.edu}

\and

\author{Aaron Boley}
\affil{Department of Astronomy, University of Florida, Gainesville, FL 32611}
\email{aaron.boley@gmail.com}

\begin{abstract}

Using 3D hydrodynamic simulations of a cooling protostellar disk undergoing gravitational instabilities (GIs), we compute the effective \citet{shakura1973} $\alpha$ due to gravitational stresses and compare them to predictions from an analytic local theory for thin disks by \citet{gammie2001}.  Our goal is to determine how accurately a locally defined $\alpha$ can characterize mass and angular momentum transport by GIs in disks.  We examine disks with an imposed constant global cooling time \citep{mejia2005}.  Models were computed with grids of different spatial resolution to investigate how the computed $\alpha$ is affected by numerical resolution.  

\end{abstract}

%\keywords{ }

%%%%%% INTRODUCTION %%%%%%%%%%%%%%%
\section{Introduction}

While detailed three-dimensional hydrodynamic modeling of protostellar disks can provide considerable insight into the physical and thermal states of these disks, computing resources limit such studies to snapshots spanning timescales much shorter than a disk's evolutionary lifetime.  In contrast, approaches that replace detailed physics with statistical parameters that encapsulate this physics allow for calculations covering much longer timeframes.  Both approaches can play important roles in broadening our understanding of these complex nonlinear systems, provided a connection between the detailed physics and statistical parameters can be demonstrated and understood. In what follows, we examine the applicability of an $\alpha$-disk formulation \citep{shakura1973} to  the
evolution of protostellar disks subject to gravitational instabilities. 

Gravitational instabilities (GIs) play an important, {\bf SM: I'm not sure this is true in general. Perhaps reword to ``GIs can play an important role...and may dominate the structure for certain phases of evolution''}and at times dominant, role in driving the structural and thermal
evolution of  protostellar disks \citep[for a review, see][]{durisen2005}.
%In particular, torques arising from GIs provide a very effective means of redistributing matter and 
%angular momentum within these disks 
%\citep{Lod04, Lod05, Vor05, Vor06, Bol09}.  
Thermal processes play the primary role in regulating the amplitude and outcome of these instabilities \citep{pickett1998,pickett2000,mejia2005,nelson1998,nelson2000a}. A a disk's susceptibility to GIs is parameterized by the Toomre $Q$-parameter \citep{toomre1981};  $Q = c_s \kappa / \pi G \Sigma$, where $c_s$ is the sound speed, $\kappa$ is the epicyclic frequency ($\sim$ the rotation frequency 
$\Omega$ in a nearly Keplerian disk), and $\Sigma$ is the disk surface mass density. For $Q \al$  1.5 to 1.7,  small density perturbations in a disk grow exponentially on a time scale comparable to the rotation period. These perturbations manifest themselves as multi-arm spirals with a predominantly trailing pattern that transports angular momentum outward by gravitational torques \citep[e.g.]{larson1984,boss1984,durisen1986,papaloizou1991, laughlin1994,nelson1998,pickett1998}. The long range nature of these torques cause nonlinear
spiral structures to develop over a broad range of radii \citep{laughlin1998,nelson1998,nelson2000a,pickett1998, pickett2000,pickett2003}. {\bf SM: not sure of the meaning here. A low Q is what causes GIs to develop over a broad range of radii. Perhaps you mean the long range nature can cause coherent spiral structures to form with large radial extent?}

Numerous studies have shown that thermal physics control GI amplitudes by a {\it balance} of heating and cooling \citep[for example][]{tomley1991,tomley1994,pickett1998,pickett2000, pickett2003,gammie2001,boss2002,rice2003b,mejia2005,boley2006,boley2007, stamatellos2008,cossins2009}.
Heating by GIs tends to increase $c_s$, thus increasing $Q$. If disk heating persists, the instability will be suppressed once $Q$ becomes sufficiently large.  However, heating and cooling can reach a balance at nonlinear wave amplitude. GI-activity can sustain this balance  at a relatively constant,
but unstable, value of $Q$ \citep{paczynski1978, lin1981,goldreich1965},and 
the disk can exist in a state of quasi-equilibrium. In this state cooling is 
balanced by heating arising from the inward flow of matter and shocks induced by the GIs
\citep{gammie2001, lodato2004, rice2005, cossins2009, vorobyov2010}.

Fully nonlinear hydrodynamic simulations of low-$Q$ disks show that multiple
modes can become unstable in the linear regime \citep[e.g.][]{nelson1998, pickett1998} {\bf additional refs?}.  Even if the initial growth is dominated by one mode, numerous modes usually appear in the nonlinear regime. As a consequence, if a disk achieves a quasi-steady balance between heating and cooling, then there tends to be power at all $m$-values resolved by a given numerical method \citep{mejia2005,boley2006}.  
In this  {\it asymptotic state}, GI-active disks develop overlapping density structures of different strengths, geometries and coherence, producing strongly nonlinear turbulence that pervades the entire disk \citep[e.g.][]{pickett2003, mejia2005,boley2006}.
\citet{gammie2001} refers to this state as {\it gravitoturbulence}. Gravitoturbulence provides a possible bridge between the detailed physics of GIs and the viscous transport parameter $\alpha$. 

\citet{shakura1973} first parameterized turbulent transport in a steady-state disk with the  parameter $\alpha$, where the source of the turbulence is unspecified. Although their original work was directed towards X-ray bright
accretion disks around black holes, this heuristic approach has proved fruitful in many astrophysical situations by allowing for analytic solutions and relatively easy numerical modeling over time-spans inaccessible to detailed modeling. In the $\alpha$ disk formalism, when the accretion rate $\dot{M}$ is constant in both time and radius, $\dot{M} = 3\pi\alpha c_s \Sigma H$ where $c_s$ is the sound speed, $\Sigma$ the surface density, and $H$ the disk scale height. The $\alpha$ formalism makes no assumption about the nature of the angular momentum transport, only that {\it heating and cooling are dominated by local processes}{\bf SM: This is not my understanding of the $\alpha$ formalism, I don't think it says anything about cooling. It also means that ang mom transport happens locally and that this causes local heating}; the validity of an $\alpha$-disk picture requires self-consistency of local energy dissipation. 

Dissipative locality {\bf SM: need to explain what is meant by this more precisely} in GI-active disks has not been rigorously tested and remains an open question. 
Several studies support the idea that GI transport in real disks is, in many important respects, an intrinsically global phenomenon and cannot be properly treated by a local $\alpha$-like prescription \citep{laughlin1996, balbus1999, lodato2005, mejia2005, boley2006, cai2008}.  This almost certainly applies to the embedded phase of both high and low-mass stars when infall from the collapsing protostellar cloud feeds mass into the disk at a rapid
rate \citep[e.g.]{yorke1993, laughlin1994, yorke1999,vorobyov2005,vorobyov2006}, and disks observed in the embedded phase can be quite massive relative to their central stars \citep[e.g.][]{osorio2003}. 
However, despite the long-distance nature of gravitational torques, theoretical and numerical results suggest that in protostellar disks where heating and cooling are in balance it may be valid to represent GI-induced transport by an {\it effective viscosity}, a local process \citep[e.g.][]{gammie2001, rice2005, lodato2004}. Indeed, \citet{gammie2001} derived an effective alpha for the case of a thin 2D gravitationally unstable disk where the heating due to local turbulent dissipation caused by GIs was balanced by the local cooling rate.  Modifications to this derivation have since been proposed by {\bf (Tom: include other derivations and refs)}.   Recent simulations \citep[e.g.]{cossins2009, vorobyov2010} suggest that, at least in the case of imposed cooling, $\alpha_{\rm{eff}}$ roughly converges to the value predicted by \citeauthor{gammie2001}. 

{\bf (Tom: rework and finish this paragraph, or toss it)}
Three possibilities exist: 1) GI-transport is predominately a long-range effect and thus not 
amenable to an $\alpha$-disk representation, 2) GI-transport can be well represented as
a short-range effect and thus is directly amenable to an $\alpha$-disk representation,
If long-range energy transport and/or angular momentum transport are negligible,  or if
the long-range transport could be shown to represent a known fraction of the 
***. 
Identifying the principal angular momentum transport mechanism
and how it behaves is critical to understanding whether disks can be described by an alpha disk model.

{\bf (Tom: add text concerning need for convergence approach)}
Here we report the results of a convergence study designed to determine whether GI-active disks, chaotic by nature, achieve statistical equilibrium; if so, to what level do these disks approximate the gravitoturbulence of \citet{gammie2001} and with what distribution of $\alpha_{eff}$; and to characterize how the GI cascades from low-order to high-order modes. To this end, we follow the evolution of a GI-active disk, subject to constant cooling, in its asymptotic state, where cooling and heating are essentially balanced.  Simulations are performed using four different resolutions in the azimuthal direction.

{\bf (Tom: add paragraph providing outline of paper)}

%Nonlinear evolutions of disks undergoing GIs, even local thin-disk treatments, 
%\citep[e.g.,][]{Joh03} 
%are sufficiently complex that they must be done numerically,  
%typically using a finite element approach, where
%a set of coupled equations is solved simultaneously on a computational grid.  
%Computational constraints prohibit a full examination of GI unstable disks over 
%the life of the disk.
%Indeed, existing computational resources limit 
%detailed three-dimensional 
%simulations of disk evolution to tens of
%outer disk rotations.  
%We cannot yet follow the entire dynamical evolution of a 
%protoplanetary disk nor
%model all the radiative and particle processes of interest at once in full 3D. However, 
%by following GI simulations to their asymptotic states, when these exist, we
%can extrapolate the consequences over much longer times (e.g., Boley et al 2006).

%For 
%grids of a fixed {\it physical} size,
%grids with more elements provide finer meshes, which typically produce a more accurate solution. 
%However, this increased accuracy if offset by temporal sacrifices.
%The number of equations to be solved at each timestep 
%increases as the grid size is increased.  At the same time, the Courant condition 
%forces these timesteps to be shorter in duration.  A mesh convergence study 
%provides a common approach to balancing accuracy and computing resources.
%In the following sections we follow the evolution of a disk subject to GUs
%using four different
%resolutions in the azimuthal direction.  The number of azimuthal grid points used in simulations
%may be particularly important, because higher resolution 
%in azimuth allows GI power to spread to higher-order modes 
%that behave more locally. We test this by examining the power in various 
%modes to see how they are affected by the choice of grid.

\section{Hydrodynamical Simulations}

Simulations were conducted with the three-dimensional hydrodynamics code used by the Indiana University Hydrodynamics Group in several previous studies \citep{pickett1998, pickett2000, mejiaphd2004, mejia2005, boley2006, cai2008}.  This code uses a second-order, explicit, Eulerian scheme to solve Poisson's equation, an equation of state, and the equations of hydrodynamics in conservative form on a uniform cylindrical grid.  The code assumes mirror symmetry about the equatorial plane. It includes both self-gravity and artificial viscosity; the latter serves to 
mediate shocks. Source and flux terms \citep{norman1986} are computed separately in an explicit, second-order time integration \citep{albada1982,christodoulou1991,yangphd1992}.

We assume an equation of state $P = (\gamma -1)\epsilon$, where $P$ is the pressure, $\epsilon$ is the internal energy density, and $\gamma$ is the ratio of specific heats, given here by $\gamma = 5/3$.  The model disks are cooled by decreasing their internal energy according to the prescription $d\epsilon/dt = \epsilon / t_{cool}$, where $t_{cool}$ is the global cooling time in Outer Rotation Periods (ORP).  For the grid resolutions used here, one ORP is defined as the rotational period at radial zone 200.

%\subsection{Models}

Because this is a polytropic disk with an idealized cooling time, only the ratio of disk to stellar mass, given here by 0.153, is specified.  In what follows, quantities specified with real physical units are derived assuming a 1 $M_\odot$ central star and a 0.153  $M_\odot$ disk. In this case, the ORP is defined at $r = 33$ AU and one ORP $\approx$ 303 years.  Models have $t_{cool} = 2$ ORP. The disk has an initial surface density $\Sigma \sim r^{ -1}$, and initial inner and outer radii of 2.3 and 40 AU, respectively.  

Initial conditions for the disk's structure and thermodynamic state were set using an equilibrium star plus disk model generated using a modified \citet{hachisu1986} self-consistent field (SCF) relaxation method \citep{pickett1996,pickett2003,mejiaphd2004,mejia2005,cai2006}. Here  the density and angular momentum distributions are iteratively solved, using the specified equation of state,  until convergence is achieved. This  equilibrium disk was then seeded with small 0.01\% $\delta\rho/\rho$ random perturbations to allow GIs to grow.  The resultant marginally unstable disk served as the initial disk for the simulation.

The initially unstable disk model passed through several phases of evolution \citep[see also][]{pickett2003, mejia2005}.  The initial disk is unstable to the growth of axisymmetric structures. During this axisymmetric phase, 
which lasts several ORPs, the disk cools and contracts, the contraction is small radially but dramatic in the vertical direction. Around 3-4 ORPs the instabilities begin grow to non-linear amplitudes. When this condition is met, the disk undergoes a strong burst of GIs that predominantly manifest themselves in one or two discrete global spiral modes.  The disk expands violently during the burst phase, producing a significant rearrangement of the disk's mass distribution on a timescale of a few ORPs. This is then followed by a period of several ORPs where heating temporarily washes out some of the nonaxisymmetry in the disk.  Finally the disk settles into a quasi-steady, long-lived asymptotic state of sustained GI-activity over a large part of the disk, with an overall balance of heating and cooling.

Simulations were run with four different angular resolutions, with grid sizes in $(r,\phi,z)$ of  $(j_{max}, k_{max}, l_{max}) =$ (512,64,64), (512,64,128), (512,64,256), and (512,64,512). Because of the large computational overhead associated with starting each simulation from $t = 0$, the $l_{max} =$ 128 simulation was run through the axisymmetric, burst and transition phases. Simulations using $l_{max} =$ 64, 256, and 512 all begin by interpolating the $l_{max} =$ 128 simulation to higher or lower azimuthal resolution at 9.6 ORPs, near the end of the transition
phase. Although this strategy saves a large amount of computational time, it limits the scope of the analysis to the asymptotic phase and it introduces a perturbation to the disk which takes several ORPs to subside. By 12 ORPs, the perturbation has dissipated and all four disks have transitioned to the asymptotic state.  All simulations were 
run through $\sim 18$ ORPs.

Figure \ref{Final_Q} shows $Q$, as a function of radius, at $t \sim$ 18 ORP, while table \ref{tbl:ams} lists $Q_{avg}$, the value of $Q$ of for the asymptotic disk, time-averaged of 12 -- 19 ORP and spatially averaged over 10 -- 40 AU, as a function of radius for each resolution,  

%%%%%%%%%%%% RESULTS & DISCUSSION %%%%%%%%%%%%%%%%

\section{Results and Discussion}
	
In the asymptotic state, GI-active disks develop complex density structures arising from the superposition of
all modes present in the disk.  Since the number of modes accessible to the disk is a function of the azimuthal resolution, we expect that higher resolution simulations will manifest more power in higher-order modes 
that behave more locally. We test this by examining the power in various modes to see how they are affected by the choice of grid.

\subsection{Modes}
	
Figure \ref{fig:DensityPlots}, shows midplane and radial densities  at $t \sim 18$ ORPs for the $l_{max} =$ 64, 128, 256, and 512 simulations. The lower resolution (smaller $l_{max}$) simulations display longer wavelength spiral structures, with larger amplitudes and more coherence than those seen in higher resolution simulations. Higher angular resolution allows GI power to spread to higher-order azimuthal modes, as demonstrated by a Fourier decomposition of the density. The strengths of these modes are given by their global Fourier amplitudes,

\begin{equation}
A_m = \frac{(a_m^2 + b_m^2)^{1/2}}{\pi\int\rho_o \varpi {\rm d}\varpi {\rm d}z},
\end{equation}
where
\begin{subequations}
\begin{align}
a_m &= \int \rho \cos(m\phi)\varpi {\rm d}\varpi {\rm d}z {\rm d}\phi,\\
b_m &=\int \rho \sin(m\phi)\varpi {\rm d}\varpi {\rm d}z {\rm d}\phi.
\end{align}
\end{subequations}


Here  $\rho_o$ is the axisymmetric (zero-order)  component of the density. In what follows, $A_m$ is averaged over time to suppress fluctuations on the dynamic time scale due to the chaotic nature of GIs in the asymptotic  phase.
In particular, $\langle A_{tot} \rangle$ will represent the power over the radial range 10 -- 40 AU, integrated from $ m = 2$ to $l_{max}/2$ and averaged over the time range 12 -- 19 ORPs. The Fourier amplitudes cannot be accurately measured at resolutions smaller than $l_{max}/2$ \citep{shannon1984}, therefore $A_m$ is limited to modes $m \le l_{max}/2$.  
Because the central star is artificially fixed to the grid center, the $m = 1$ mode cannot be be realistically treated; it is excluded from $\langle A_{tot} \rangle$.

{\bf (Note to all: Throughout the paper I use integrals in places where we 
formally have used summations.  I note that this is a common practice, though
not strictly correct.  I find the use of integrals more elegant, but am willing to
revert to summations if people push for it.)}

Previous studies have shown that disks in the asymptotic state have power at all resolvable $m$-values  \citep[e.g.][]{mejia2005,boley2006}.  Table \ref{tbl:ams} lists $\langle A_{tot} \rangle$ along with $\langle A_{2-7} \rangle  /  \langle A_{tot} \rangle$,  the fraction of the total power falling into low-order ($m = 2$ -- 7) modes, for each resolution. While the total power is approximately independent of resolution, $\langle A_{tot} \rangle = 1.90 \pm 0.07$, 
the distribution of this power shows marked differences. In particular, the fractional power in lower order modes decreases monotonically with increasing resolution.  Over three quarters of $\langle A_{tot} \rangle$ resides
in modes $m =$ 2 -- 7 for the $l_{max} = 64$ simulation, while approximately half resides in these modes when $l_{max} =$ 512.

The shift of power from lower to higher order modes as the resolution of the computational grid is increased arises from the fact that, while $\langle A_{tot} \rangle$ is approximately independent of $l_{max}$, {\bf SM: Need some explanation as to why this is the case.} the number of degrees of freedom available for this power, $l_{max}/2$, increases linearly with the resolution. Thus power is naturally spread from lower order to higher order modes as the available $m$-values increases. Indeed, even if the disk is dominated by one or two low-order modes at the time of outburst, this power cascades to higher-order modes \citep[for example][]{laughlin1997, laughlin1998,laughlin1996}.


Given the resolution dependence of how power is distributed,  we must examine what resolution is sufficient to properly capture the evolution of these disks. Table \ref{tbl:ams} shows that while the fractional power in lower order modes decreases with each doubling of $l_{max}$, the amount of change in going from $l_{max} = 256$ to 512 
is significantly less than the change going from $l_{max} = 64$ to 128, with a strong suggestion that this diminution will not proceed much further with higher azimuthal resolution. Indeed, a  Richardson extrapolation {\bf (add reference) SM: suggest Press et. al} \citep{press1992} of the tabulated values of  $\langle A_{2-7} \rangle / \langle A_{tot} \rangle$ suggests a limiting value of  $\langle A_{2-7} \rangle / \langle A_{tot} \rangle \sim$ 0.43 for increasingly higher resolutions. This is displayed graphically in figure 4, where the fractional power in lower order modes is plotted against the  normalized grid 
spacing for each value of $l_{max}$. The grid spacing  is normalized such that spacing of the $l_{max}$ = 512 grid
equals unity.  Convergence occurs as the normalized spacing goes to zero. The Richardson extrapolation value  of $\langle A_{2-7}\rangle  / \langle A_{tot}\rangle  = 0.43$ is plotted at a normalized grid spacing  equal to zero. More than half of the modal power in a converged disk resides in higher-order modes, modes that help define a state of gravitoturbulence. {\bf SM: I'm not sure this is accurate, gravitoturbulence isn't really turbulence (at least as it's normally defined) and the high order modes aren't really turbulence either} The fact that the $l_{max} =$ 512 simulation has $\langle A_{2-7} \rangle / \langle A_{tot} \rangle$ near the convergence value lends some confidence in the use of the $l_{max} =$ 512 model for the studie of effective $\alpha$. 

\subsection{Gravitational Torques}

Coherent density structures in a disk produce gravitational torques. Low-order modes in $\Delta\rho / \rho$  have longer wavelengths and, if coherent in radius, have longer lever arms to produce torques. On the other hand, high-order modes have relatively short wavelengths which, particularly if they lack coherence, produce more localized effects. The high-order modes tend to produce pockets of local GI activity, which can cancel each other
over the whole disk, {\bf SM: I think there's a sentence missing here explaining that higher resolution leads to more of the amplitude being concentrated in high order modes, while the total amplitude stays the same.} leading to the expectation that  gravitational torques in the disk will decrease as $l_{max}$ is increased. This can be tested by examining the torque contributions from  various Fourier modes.

The torque {\bf C} on a cylindrical surface of the disk at radius $\varpi$ can be obtained by integrating the stress tensor $T$ over the surface of the cylinder \citep{lyndenbell1972}, i.e.,
\begin{equation}
{\rm {\bf C}} = \int {\bf r} \times T \cdot {\rm d}S.
\end{equation}
If the stress tensor includes only gravitational stresses, the surface integral can be replaced with the volume integral
\begin{equation}
{\rm \bf C} = \int \rho {\bf r} \times \nabla \Phi {\rm d}V,
\end{equation}
where $\Phi$ is the gravitational potential.  Here we are interested only in the $z$-component of torque,  
\begin{equation}
\label{torque1}
{\rm \bf C}_Z = \int \rho \frac{\partial\Phi} {\partial \phi} {\rm d}V,
\end{equation}
as only this component drives mass and angular momentum transfer. The torque contribution from each mode $m$ can be calculated by replacing $\rho$ in Equation (5) with the density distribution reconstructed from a single Fourier component
\begin{equation}
\rho_m = a_{\phi m} \cos(m\phi) + b_{\phi m} \sin(m\phi),
\end{equation}
where $a_{\phi m} =\slantfrac{1}{\pi} \int \rho \cos(m\phi){\rm d}\phi$ and $b_{\phi m} =\slantfrac{1}{\pi} \int \rho \sin(m\phi){\rm d}\phi$, and only the gravitational potential produced by the mass distribution given by $\rho_m$ is included in $\Phi$.

Figure 5 {\bf SM: Two things about this figure. Are you sure you removed the m=1? You would have had to have written your own software or significantly changed what I gave you to do this. Also I think the amplitude of the torques are not right if we're really considering a 1$M_\odot$ star. These numbers look like the ones for a 0.5$M_\odot$ star} displays $\langle {\rm \bf C}_{Ztot} \rangle$, the time-averaged gravitational torque summed over all disk modes (excluding $m=1$), along with the torques summed over low-order modes, $\langle {\rm \bf C}_{Z(2-n)} \rangle$, where $n = $ 2, 3, 4, and 5, as a function of radius and azimuthal resolution. 
{\bf (Note to all: the version of figure 5 included with this draft  shows the torques
time-averaged over for four different time-frames.  I did this because I wanted to see if the disk was 
"settling down" with time.  I think the figure suggests  that the 12-14 ORP disk is not
yet in an asymptotic state - or further from it than at later times - I'm not sure this is
a significant difference.  Looking forward to figure 7, it can be argued that the
disk is reasonably settled, in a statistical sense, shortly after 12 ORP.
Dick has told me he only believes the 12-18 ORP result - Dick, I hope 
I'm not misquoting you).  
} 
The scale of the $y$-axis in each subplot is the same, with the exception of the $l_{max} = 64$ simulation. 
In this case the scale is doubled.  Two features are immediately apparent: 1) the total gravitational torque is smaller for higher resolution simulations, and 2) the total torque is dominated by low-order modes for all azimuthal resolutions. Both reflect the fact that a while higher fraction of the total power in larger $l_{max}$ simulations is in higher order modes, these modes make only minimal contributions to the gravitational torque. As resolution increases, non-axisymmetric power is shifted from low-order global modes, which produce relatively large gravitational torques, to high-order local modes, which tend to cancel each other out when integrated over the whole disk. 

\subsection{Effective Alphas}

An $\alpha$-framework treats disks as essentially two-dimensional objects subject to local energy dissipation.  Therefore, in order to compare our results with predicted and published results, we must extract two-dimensional information from our three-dimensional disks.  Following \citet{lodato2004}, 
\begin{equation}
\alpha_{\rm{eff}}(\varpi) = \left| \frac{{\rm d} \ln \Omega}{{\rm d} \ln \varpi} \right|^{-1} 
\frac {\langle T_{\varpi\phi}^{grav}\rangle + \langle T_{\varpi\phi}^{Reyn}\rangle}
{\Sigma c_s^2},
\end{equation}
where $\Omega$ is the azimuthally averaged rotation speed, $T_{\varpi\phi}^{grav}$ and $T_{\varpi\phi}^{Reyn}$ represent the azimuthally averaged $\varpi,\phi$-component of the specific (per unit mass) gravitational 
(or ``Newton'') and hydrodynamic (or ``Reynolds'') stress tensors, respectively, $\Sigma$ the surface density,
and $c_s$ the azimuthally averaged midplane sound speed.  Brackets surrounding the stress terms indicate that these are integrated over the $z$-direction. The denominator $\Sigma c_c^2$ was evaluated as $\int_z \rho c_c^2 {\rm d}z$. {\bf SM: I need to rewrite this entire paragraph}
 
Numerical uncertainties render the Reynolds stress difficult to determine. Fortunately, this difficulty is mitigated by the fact that previous studies have shown the magnitude of the Reynolds stress to be small relative to the gravitational stress (references?).
Thus it can be safely neglected in Equation (7).

The gravitational torque from equation \ref{torque1} can be used to determine the
gravitational stress by
\begin{equation}
\begin{split}
 T_{\varpi\phi}^{grav}\left( \varpi \right) & = \frac{1}{2\pi\varpi^2}C_Z\\
                                & = - \frac{1}{2\pi\varpi^2} \int \rho\frac{\partial\Phi}{\partial\phi} {\rm d}V.
\end{split}
\end{equation}
Figure 6 shows the effective $\alpha$s for the four different grid resolutions, time averaged from 12 to 19 ORPs.{\bf SM: is this 18 or 19 ORPs?} For comparison, the predicted values by \citet{gammie2001} for the strongly self-gravitating limit (lower curve) and the non-self-gravitating limit (upper curve) are shown. Note that predictions from {\bf ?? is this \citep{gammie2001}?} are based upon a thin
disk approximation, where the local balance of heating and cooling dominates the energetics.

In an $\alpha$ disk formalism the mechanism giving rise to the transport of angular momentum also acts to dissipate energy. Lodato \& Rice (2005) postulated that a disk in a state of gravitoturbulence would have locally balanced heating and cooling rates. If the heating were due to local dissipation, then $\alpha$ and $t_{cool}$ are related by
\begin{equation}
\alpha = \left| \frac{{\rm d}\ln \Omega}{{\rm d} \ln R} \right|^{-2}
\frac{1}{\gamma(\gamma - 1) t_{cool}\Omega}.
\end{equation}
\citet{rice2005} concluded that for a given $t_{cool}$  there exists some maximum stress quantified by $\alpha_{max}$, which if exceeded would cause a disk to fragment. The horizontal dot dashed line in figure 6 indicates the $\alpha_{max} \approx 0.06$ found by \citeauthor{rice2005}. \citet{clarke2007} have revised this limit to $\alpha_{max} = 0.12$ for cases where $t_{cool}$  varies with time. 

Figure 7 shows the time-variation of $\alpha$, normalized to its value at $t = 18$ ORP), as a function of radius and  time.  This plot suggests that the disk has not yet achieved statistical equilibrium until $\sim $14 ORP {\bf (Everyone:  What do you think?  We are using averages over12-18 ORP,
are we comfortable with those limits?)}

\begin{center}
\line(1,0){250}
\end{center}

{\bf From this point on the draft is unfinished; the text that follows is extracted from 
Scott's Chapter 3, with little to no modification, so for now consider it as filler.   
I am still working on the rest of this section, which I expect to be the concluding section
of the paper (excluding a summary).  Rather than further delay getting
the draft to you, I am sending it as is.   I will be adding text and incORPorating results from a 
few recent papers and will send the rest soon.}

\begin{center}
\line(1,0){250}
\end{center}


As the azimuthal resolution increases, the effective $\alpha$ in the GI active region decreases. This is due to the decrease in the total gravitational torque, as illustrated in figure 5. This decrease in gravitational torque is due, in turn, to the shift of non-axisymmetric power from low-order global modes, which produce large torques, to high-order local modes, which can cancel each other out when integrated over the whole disk. In fact, the decrease in the ratio of low-order power to the total power $\langle A_{2-7}\rangle / \langle A_\Sigma \rangle$ closely tracks the decrease in the time averaged $\alpha$ averaged from 10 to 40 AU, which can be seen in table \ref{tbl:ams} as the $\alpha_{avg}$ value. Although the average $\alpha$ values have not converged to a single value for my highest two resolutions, the percentage change continues to decrease. In addition, figure 6 shows that the $l_{max} =$ 256
and 512 simulations have qualitatively similar time averaged profiles over a large range of disk radii. These profiles also fall within the predicted Gammie curves, with tendency toward the lower (strongly self-gravitating) curve. Indeed the $\alpha_{avg}$ value of 0.020 for the $l_{max} = $ 512 simulation is squarely between the $\alpha_{avg}$ values for the Gammie curves 0.014 (strongly self-gravitating) and 0.027 (non-self-gravitating).

Although the effective $\alpha$ curves do correspond to the Gammie prediction, the results from these simulations differ from Gammie's in one significant way. The local shearing box simulations performed by \citet{gammie2001} show that the total effective $\alpha$ is composed of a gravitational stress component and a Reynolds stress component that are nearly equal (see Gammie (2001) figure 3). However, all of the analyses
presented here only consider the gravitational stress. When I attempt to measure
the Reynolds stress, I find it is quite small compared to the gravitational stress. As
noted previously, the Reynolds stress is very hard to determine in global simulations.
However, the prediction made by Gammie relies only on the idea that heating from
GI activity balances cooling. So the exact mix of stresses contributing to the total
amount of heating could be different depending on the type of simulation, the physics
considered in the simulation, and the simulation parameters.


\section{Summary}

When considering the effect of azimuthal resolution on the asymptotic behavior
of a GI unstable disk I found that without sufficient azimuthal resolution of at least
$l_{max} =$ 512 gravitational torques and effective gravitational $\alpha$
will be underestimated
for disk with a constant global cooling rate. I also found that, when properly resolved,
more non-axisymmetric power propagates to higher order modes, and the amplitude
of low-order modes decreases. This results in a decrease of the gravitational torque
and effective gravitational $\alpha$. For $l_{max} =$ 512 the effective gravitational 
$\alpha$ measured
is in very good agreement with the prediction of Gammie (2001). Gammie's 
prediction follows from an argument of balance of GI heating and cooling, resulting in
``gravitoturbulence''. However, his detailed simulations found that the torques and
heating in the disk were due in equal parts to gravitational stresses and Reynolds
stresses, this is in contrast to my measurements which only consider gravitational
stresses.


\acknowledgements

\bibliographystyle{apj}
\bibliography{general}


\newpage



% \begin{deluxetable}{cccc} 
% \tablecolumns{4} 
% \tablewidth{0pc} 
% \tablecaption{Effective Alphas} 
% \tablehead{ 
% \colhead{}    &  \multicolumn{3}{c}{Non-shell Stars} &   \colhead{}   & 
%\multicolumn{3}{c}{Shell Stars} \\ 
%\cline{2-4} \cline{6-8} \\ 
%\cline{2-8} \\ 
% \colhead{Azimuthal Zones} & \colhead{$\alpha$}   & \colhead{$A_{tot}$}    
% & \colhead{$A_{2-5} / A_{tot}$} \\ 
%\colhead{} & \colhead{} &\colhead{} &  \colhead{angle} & \colhead{$10^{-28}$ ergs /}  \\
%\colhead{} & \colhead{} &\colhead{} &  \colhead{angle} 
%&  \multicolumn{2}{c}{$(10^{-28}$ ergs / cm$^3$ s sr)}  \\
% } 
% \startdata
% 64	 &    0.0587	 &        2.02 	 & 0.752	      \\
% 128	   &    0.0402	  &       1.99  	 & 0.608      \\
% 256	  &     0.0237	 &        1.87	 & 0.535     \\
% 512	  &     0.0241	 &        2.28	 & 0.467     \\
% \enddata 
% \tablecomments{
% Table 1. The $\alpha$ of figure 1, averaged over the radial region from 10 -- 40 AU, tabulated for different 
% azimuthal resolutions, along with amplitudes of Fourier components for non-axisymmetric density structures. 
% The radially averaged $\alpha$ predicted by Gammie (2001) are tablulated for comparison.
% }
% \end{deluxetable}

\begin{deluxetable}{cccccc}
\tablecolumns{5} 
\tablewidth{0pc} 
\tablecaption{$Q$, Fourier Amplitudes, and $\alpha$} 
\tablehead{ 
   \colhead{$l_{max}$} 
& \colhead{$Q_{avg}$}   
& \colhead{$\langle A_{tot} \rangle$}    
& \colhead{$\langle A_{2-7} \rangle / \langle A_{tot} \rangle$} 
& \colhead{$\alpha_{avg}$} \\ 
} 
\startdata
64   & 1.26 & 1.96 &  0.76 & 0.069 \\
128 & 1.24 & 1.91 &  0.62 & 0.044 \\
256 & 1.39 & 1.83 &  0.54 & 0.026 \\
512 & 1.48 & 1.95 &  0.49 & 0.020 \\
\enddata 
\tablecomments{
All quantities are time-averaged over 12 -- 19 ORP.
$Q_{avg}$ and $\alpha_{avg}$ are also spatially
averaged over 10 - 40 AU.
}
\label{tbl:ams}
\end{deluxetable}
\newpage

\begin{figure}\label{Final_Q}
\plotone{figures/qcompareres.eps}
\caption{Toomre Q distribution versus radius for each of the four resolutions at t $\sim$ 18 ORPs}
\end{figure}

\begin{figure}\label{fig:DensityPlots}
%\plotone{figures/Figure_1.eps}
\caption
{
Midplane disk densities at $t \sim 18$ ORPs for simulations with 64, 128, 256 and 512 azimuthal grid elements.
The simulations differ only in the number of  azimuthal zones used in the calculations.  Densities out of the plane, along an azimuthal cut through the disk, are displayed below the face-on views.  The color scale is  logarithmic, and axis units are AU. 
}
\end{figure}
\newpage

\begin{figure}\label{fig:Am_vs_log_m}
%\plotone{figures/Figure_3.eps}
\caption
{
The logarithm of the amplitudes for the global Fourier components of the non-axisymmetric density structure plotted for different $m$-values. Notice that the components with the highest global amplitudes are those with low $m$-values.  
}
\end{figure}

\begin{figure}
%\plotone{figures/Power_Extrapolation.eps}
\caption
{
Convergence of fractional power.  The $x$-axis represents
the azimuthal ``spacing'' between grid points,  normalized such that spacing of the $l_{max}$ = 512 grid
equals unity.
The $y$-axis displays the fraction of the total power in the low-order ($m = $ 2--7) Fourier modes. 
Also plotted is the Richardson Extrapolation for the asymptotic value  
$\langle A_{2-7} \rangle / \langle A_{tot} \rangle = 0.43$ as the grid spacing goes to zero.  
}
\end{figure}

\begin{figure}
%\plotone{figures/Torque_Array.eps}
\caption
{
Torque Array}
\end{figure}

%\begin{figure}
%\plotone{figures/Mode_CascadeRatios.pdf}
%\caption
%{
%Mode Cascade Ratios .  Note that by the end of the time frame shown, there is rough parity between 
%torques arising from tower order ($m = 2$ -- 5) mades and torques from higher order modes.}
%\end{figure}


\begin{figure}
\label{fig:alpha_v_radius}
%\plotone{figures/Figure_2.eps}
\caption
{Effective $\alpha$'s for the $t_{cool} =$ 2 ORP, $\gamma = 5/3$, $\Sigma \sim r^{ -1}$ disk, averaged between 12 and 18 ORPs, for four different different azimuthal grid resolutions, given by the number of azimuthal grid-points Lmax. The parallel dashed lines show the predictions of Gammie (2001) for $\alpha$ (upper: strongly self-gravitating limit, lower: non-self-gravitating limit), while the dashed line at $\log(\alpha) = -1.22$ represents the maximum stress a disk can maintain without fragmenting (Rice et al. 2005) . 
}
\end{figure}
\newpage

\begin{figure}
%\plotone{figures/Alpha_variations_stack512.eps}
\caption
{Alpha variations.}
\end{figure}


\end{document}

